\documentclass[a4paper, twoside, italian, 11pt]{book}

\usepackage{amsfonts}  % for mathbb
\usepackage{amsmath}   % for implies
\usepackage{indentfirst}
\usepackage{amssymb}   % for emptyset
\usepackage{accents}   % for circle above characters (mathring)

\newcommand{\braces}[1] {\left \{ #1 \right \}}
\newcommand{\abs}[1] {\left | #1 \right |}
\newcommand{\overbar}[1] {\mkern 1.5mu\overline{\mkern-1.5mu#1\mkern-1.5mu}\mkern 1.5mu}

\DeclareMathOperator{\dom}{dom}
\DeclareMathOperator{\codom}{codom}
\DeclareMathOperator{\Ima}{Im}
\DeclareMathOperator{\im}{im}
\DeclareMathOperator{\graf}{graf}

\newcommand{\N}{\mathbb{N}}
\newcommand{\Z}{\mathbb{Z}}
\newcommand{\Q}{\mathbb{Q}}
\newcommand{\K}{\mathbb{K}}
\newcommand{\R}{\mathbb{R}}
\newcommand{\E}{\mathbb{E}}

\let\emptyset\varnothing

\begin{document}

\pagestyle{headings}

\frontmatter

\begin{titlepage}
	\begin{center}
		{\huge \bfseries Appunti di Analisi Matematica I \\}
		\vspace{1.5cm}
		{\Large \bfseries Ettore Forigo}
	\end{center}
\end{titlepage}

\mainmatter

\chapter{}



\section{Insiemi Famosi}

\noindent
$\N$ = Numeri Naturali $= \braces{0, 1, 2, 3, ...}$ \\
$\Z$ = Numeri Interi \\
$\Q$ = Numeri Razionali \\

\noindent
$\N \subset \Z \subset \Q$ \\

\noindent
Su $\Q$ è definita una relazione d'ordine totale  ($\leq$) \\

\noindent
Gli insiemi con relazioni d'ordine totale si chiamano totalmente ordinati.



\section{Dimostrazioni}


\subsection{Componenti delle Dimostrazioni}

\noindent
I 3 termini seguenti, in ordine di importanza crescente, sono abbastanza sinonimi; cambia solo l'importanza nell'ambito dell'esposizione di una teoria formale:


\subsubsection{Proposizione}

\noindent
Proposizione logica che può assumere valori di verità.


\subsubsection{Lemma}

\noindent
Teorema ausiliario o intermedio utilizzato nella dimostrazione di un'altra proposizione o teorema.


\subsubsection{Teorema}

\noindent
Congettura (proposizione) dimostrata.


\subsubsection{Corollario}

\noindent
Proposizione dimostrata a partire da un teorema.


\subsection{Forma dei Teoremi}

\noindent
$A \implies B$ \\

\noindent
Dove la proposizione $A$ è detta ipotesi e la proposizione $B$ è detta tesi.


\subsection{Implicazioni}

\noindent
$P \implies Q$ \\

\noindent
Dove la proposizione $P$ è detta antecedente e la proposizione $Q$ è detta conseguente (anche come sostantivi maschili).


\subsection{Dimostrazione di una Implicazione}

\noindent
Si assume l'antecedente (o premessa) e si dimostra il conseguente.


\subsection{Dimostrazione per Assurdo}

\noindent
Si suppone l'ipotesi e per assurdo si suppone il contrario della tesi, e si trova una contraddizione.



\section{Definizione del Principio di Induzione}

\indent
$P(n_0) \land (P(n) \implies P(n + 1)) \implies \forall n \in \N \ldotp P(n)$ \\

\noindent
Il caso base nell'induzione può essere anche un numero $\neq 0$.



\section{Campo Ordinato dei Razionali}
\noindent
$(\Q, \leq)$ formano un Campo Ordinato.


\section{Definizione di Completezza di un Campo}

\noindent
Un campo totalmente ordinato $(\K, \leq)$ si dice completo se vale il seguente \textbf{assioma di completezza} (Assioma di Dedekin): \\

$A \subseteq \K, B \subseteq \K, A \neq \emptyset, B \neq \emptyset$ \\

$\forall x \in A, \forall y \in B \ldotp x \leq y \implies \exists c \in \K : \forall x \in A, \forall y \in B \ldotp x \leq c \leq y$ \\

\noindent
$c$ è chiamato elemento separatore tra gli insiemi A e B. \\

\noindent
Il campo $(\Q, \leq)$ è totalmente ordinato ma non completo.


\section{Definizione dei Numeri Reali}

\noindent
$\R$ è una estensione di $\Q$ tale che il campo $(\R, \leq)$ è totalmente ordinato e completo.


\subsection{Interpretazione Geometrica}

\noindent
Ogni numero reale può essere univocamente associato ad un punto della retta reale e viceversa



\section{Definizione Numeri Irrazionali}

\noindent
$\R \setminus \Q$ = Numeri Irrazionali



\section{Definizione di Massimi e Minimi}

$\E \subseteq \R, \E \neq \emptyset$ \\

$\exists a \in \E : \forall x \in \E \ldotp a \leq x \implies a$ è un minimo di $\E$ \\

$\exists b \in \E : \forall x \in \E \ldotp x \leq b \implies b$ è un massimo di $\E$ \\

$\min(\E) = a$ \\
\indent
$\max(\E) = b$ \\

\noindent
Esistono insiemi limitati che non ammettono né massimo né minimo. \\

$\E = \braces{x \in \R : 0 < x < 1}$


\subsection{Lemma: Unicità di min e max}

Se $\E \subseteq \R$ ammette minimo o massimo, allora è unico.


\subsubsection{Dimostrazione Unicità del Minimo}

\noindent
$a \in \E,$ $a' \in \E$ \\

\noindent
$\forall x \in \E \ldotp a \leq x \land a' \leq x$ per assurdo. \\

\noindent
Ponendo $x = a$ ottengo $a' \leq a$ \\
Ponendo $x = a'$ ottengo $a \leq a'$ \\

\noindent
Siccome devono valere entrambe, $a = a'$. Q.E.D.



\section{Definizione di Maggioranti e Minoranti}

\noindent
$\E \subseteq \R, \E \neq \emptyset$ \\

$a \in \R$ è un \textbf{minorante} di $\E$ se $\forall x \in \E \ldotp a \leq x$ \\

$b \in \R$ è un \textbf{maggiorante} di $\E$ se $\forall x \in \E \ldotp x \leq b$ \\

\noindent
\textbf{Non sono unici!} \\

\noindent
$M(\E)$ = \textbf{Insieme dei maggioranti} di $\E$\\
$m(\E)$ = \textbf{Insieme dei minoranti} di $\E$



\section{Definizione di Insieme Limitato}

\noindent
$E \subseteq \R, \E \neq \emptyset$ \\

$M(\E) \neq \emptyset \implies \E$ è \textbf{superiormente limitato} \\
\indent
$m(\E) \neq \emptyset \implies \E$ è \textbf{inferiormente limitato} \\
\indent
$M(\E) \neq \emptyset \land m(\E) \neq \emptyset \implies \E$ è \textbf{limitato}



\section{Teorema}

\noindent
$\E \subseteq \R, \E \neq \emptyset$ \\

\noindent
$\E$ è superiormente limitato $\implies M(\E)$ ammette minimo (estremo superiore di $\E$) \\

\noindent
$\E$ è inferiormente limitato $\implies m(\E)$ ammette massimo (estremo inferiore di $\E$)


\subsection{Dimostrazione}

\noindent
$\E \neq \emptyset,$ $M(\E) \neq \emptyset$ \\

\noindent
$\forall x \in \E, y \in M(\E) \ldotp x \leq y$ \\

\noindent
Quindi per l'assioma di completezza: \\

\noindent
$\exists c \in \R : \forall x \in \E, y \in M(\E) \ldotp x \leq c \leq y$ \\

\noindent
$\forall x \in \E \ldotp x \leq c \implies c \in M(\E)$ \\
$\forall y \in M(\E) \ldotp c \leq y \implies x = \min M(\E)$



\section{Definizione di Estremo Superiore e Inferiore}

\noindent
$\E$ è superiormente limitato $\implies \sup(\E) = \sup \E = \min(M(\E))$ \\

\noindent
$\E$ è inferiormente limitato $\implies \inf(\E) = \inf \E = \max(m(\E))$


\subsection{Proprietà}

\noindent
$\sup \E \in \E \implies \sup \E = \max \E$ \\
$\inf \E \in \E \implies \inf \E = \min \E$ \\
$\sup \E$ e $\inf \E$ sono unici.



\section{Caratterizzazione di sup e inf}

\noindent
$\E \subseteq \R,$ $\E \neq \emptyset,$ $\E$ superiormente limitato \\

\noindent
N.d.r. \\
Tutti gli $\varepsilon$ nelle definizioni e dimostrazioni sono da considerarsi $\in \R$ salvo diversamente specificato.


\subsection{Caratterizzazione di sup}

\noindent
$L = \sup \E \iff \forall x \in \E : x \leq L$ $\land$ $\forall \varepsilon > 0$ $\exists x \in \E : x > L - \varepsilon$


\subsection{Caratterizzazione di inf}

\noindent
$l = \inf \E \iff \forall x \in \E : l \leq x$ $\land$ $\forall \varepsilon > 0$ $\exists x \in \E : x < l + \varepsilon$



\section{Definizione di $\overbar\R$}

\noindent
Insieme dei numeri reali estesi: \\
$\overbar\R = \R$ $\cup$ $\braces{-\infty, +\infty}$


\subsection {Relazione d'ordine $\leq$ e le operazioni somma e prodotto su $\overbar\R$}


\subsubsection{Relazione $\leq$}

\noindent
$\forall x \in \overbar\R: -\infty \leq x \leq +\infty$ \\
$\forall x \in \R: -\infty < x < +\infty$


\subsection{Somma}

\noindent
$\forall x \in \R: x + \infty = +\infty$ \\
$\forall x \in \R: x + (-\infty) = -\infty$


\subsection{Prodotto}

\noindent
$\forall x > 0, x \in \R$ \\
$x \cdot (+\infty) = +\infty$ \\
$x \cdot (-\infty) = -\infty$ \\


\noindent
$\forall x < 0,$ $x \in \R$ \\
$x \cdot (+\infty) = -\infty$ \\
$x \cdot (-\infty) = +\infty$ \\

\noindent
N.B. \\
Non sono definite le operazioni: \\
$0 \cdot (\pm \infty),$ $+ \infty - \infty$



\section{Intervalli}

\noindent
$I \subseteq \overbar\R : \forall x, y \in I : x < z < y \implies z \in I$ \\
$I$ è un detto \textbf{intervallo}. \\

\noindent
$a, b \in \overbar\R, a < b$ \\


\subsection{Intervallo aperto di estremi $a$ e $b$}

\noindent
$(a, b) = ]a, b[ = \braces{ x \in \overbar\R : a < x < b}$ \\


\subsection{Intervallo semi-aperto a destra di estremi $a$ e $b$}

\noindent
$[a, b) = \braces{ x \in \overbar\R : a \leq x < b}$ \\


\subsection{Intervallo semi-aperto a sinistra di estremi $a$ e $b$}

\noindent
$(a, b] = \braces{ x \in \overbar\R : a < x \leq b}$ \\


\subsection{Intervallo chiuso di estremi $a$ e $b$}

\noindent
$[a, b] = \braces{ x \in \overbar\R : a \leq x \leq b}$ \\


\subsection{}

\noindent
$\E \subseteq \R,$ $\E \neq \emptyset,$ $M(\E) = \emptyset$ \\
$\sup \E = +\infty$ \\

\noindent
$\E \subseteq \R,$ $\E \neq \emptyset,$ $m(\E) = \emptyset$ \\
$\inf \E = -\infty$



\section{Funzioni}

\noindent
Una funzione è definita da una terna $(f, A, B)$ dove: \\

\noindent
$A \subseteq \overbar\R,$ $B \subseteq \overbar\R,$ $A \neq \emptyset,$ $B \neq \emptyset$ \\
$f$ è una legge che ad ogni elemento $x \in A$ associa univocamente un elemento $f(x) \in B$. \\

\noindent
Notazione: \\
$A = \dom(f)$ (dominio di $f$) \\
$B = \codom(f)$ (codominio di $f$) \\

\noindent
Si scrive: $f : A \rightarrow B$ \\

\noindent
N.B. \\
Il codominio $B$ non è determinato univocamente da $f$. \\
Se $B$ è codominio di $f$ e $B \subseteq C$ allora anche $C$ è codominio di $f$. \\

\noindent
Due funzioni $f_1 : A_1 \rightarrow \R$ e $f_2 : A_2 \rightarrow \R$ \\
sono uguali $\iff A_1 = A_2$ $\land$ $\forall x \in A_1 = A_2 : f_1(x) = f_2(x)$


\subsection{Definizione di Insieme Immagine}

\noindent
$f : A \rightarrow B$ \\
$\im(f) = f[A] = \Ima f= \braces{y \in B : \exists x \in A : y = f(x)}$ \\

\noindent
$\im(f) \subseteq \codom(f)$


\subsection{Definizione di Iniettività}

\noindent
Una funzione da $A$ a $B$ si dice \textbf{iniettiva} se: \\

$\forall x, x' \in A \ldotp f(x) = f(x') \implies x = x'$


\subsection{Definizione di Suriettività}

\noindent
$\im(f) = \codom(f)$


\subsubsection{Interpretazione Geometrica}

\noindent
$\forall y_0 \in \codom(f)$ la retta $y = y_0$ interseca il grafico di $f$ in almeno un punto. \\

\noindent
Equivalentemente: \\
\indent
$\forall y \in \codom(f)$ \\
\indent
$f^{-1}({y}) \neq \emptyset$ \\

\noindent
Se $f : A \rightarrow B$ non è suriettiva si può rendere suriettiva restringendo il suo codominio alla sua immagine (Troncatura). \\

\noindent
Si può restringere anche il dominio per rendere la funzione iniettiva (Restrizione). % TODO Duplicato? Riga 411


\subsection{Definizione di Biiettività}

\noindent
Una funzione si dice \textbf{biiettiva} (o biiezione, o anche corrispondenza 1 a 1 o biunivoca) se è sia iniettiva che suriettiva.



\section{Definizione di Invertibilità}

\noindent
$\forall y \in B$ $\exists! x \in A : y = f(x) \implies f : A \rightarrow B$ è \textbf{invertibile}. \\

\noindent
$f : A \rightarrow B$ è invertibile $\implies f^{-1} : \im(f) \rightarrow \dom(f)$ è la funzione inversa di f. \\

\noindent
$\forall y \in (B = \im(f)) : y = f(x) \iff x = f^{-1}(y)$ \\

\noindent
Osservazione: \\

\noindent
$\forall y \in \im(f) : y = f(f^{-1}(y))$ \\

\noindent
$f$ è invertibile $\iff f$ è biiettiva \\

\noindent
Il grafico della funzione inversa: \\
$\graf(f^{-1})$ \\
$= \braces{(y, x) \in B \times A : x = f^{-1}(y)}$ \\
$= \braces{(y, x) \in B \times A : y = f(x)}$ \\
$= \braces{(y, x) \in B \times A : (x, y) \in \graf(f)}$ \\

\noindent
$(y, x) \in \graf(f^{-1}) \iff (x, y) \in \graf(f)$ \\

\noindent
$\graf(f^{-1})$ è simmetrico di $\graf(f)$ rispetto alla retta $y = x$



\section{Definizione di Restrizione}

\noindent
$f : A \rightarrow B,$ $E \subseteq A$ \\

\noindent
$f|_E : E \rightarrow B$ \\
$f|_E(x) = f(x)$ $\forall x \in E$ \\

\noindent
$f|_E$ è chiamata \textbf{restrizione} di $f$ ad $E$. \\

\noindent
Una funzione non iniettiva si può rendere iniettiva considerandone opportune restrizioni. % TODO Duplicato? Riga 367



\section{Proprietà della Composizione di Funzioni}

\noindent
Se $f$ è invertibile, allora: \\

$\forall x \in \dom(f)\ldotp (f^{-1} \circ f)(x) = x$ \\
\indent
$\forall x \in \im(f)\ldotp (f \circ f^{-1})(x) = x$ \\
\indent
$(g \circ f)^{-1} = f^{-1} \circ g^{-1}$



\section{Nozioni di Topologia in $\R$}

\noindent
Il valore del limite di una funzione può andare oltre il dominio della funzione, ma bisogna definire delle condizioni.


\subsection{Definizione di Intorno}

\noindent
Dato $x_0 \in \R$ e dato $r > 0$ \\

$I_r(x_0) = (x_0 - r, x_0 + r)$ \\

\noindent
È chiamato l'\textbf{intorno} di centro $x_0$ e raggio $r$. \\

\noindent
Nota: \\
$x_0$ è detto ``x con zero''

\subsection{Definizione di Intorno di Infinito}

\noindent
Sia $x_0 \in \braces{-\infty, +\infty}$ e sia $a \in \R$, si chiama: \\

$(a, +\infty)$ \textbf{intorno di infinito} di estremo inferiore a \\
\indent
$(-\infty, a)$ \textbf{intorno di meno infinito} di estremo superiore a


\subsection{Definizione di Punto Interno di un Insieme}

\noindent
$A \subseteq \R,$ $x_0 \in \R$ \\

$\exists \varepsilon > 0 : I_\varepsilon(x_0) \subseteq A \implies x_0$ è \textbf{punto interno} di $A$


\subsection{Definizione di Punto di Accumulazione}

\noindent
$A \subseteq \R,$ $x_0 \in \R$ \\

$\forall \varepsilon > 0 \ldotp I_\varepsilon(x_0) \cap (A \setminus \braces{x_0}) \neq \emptyset \implies x_0$ è \textbf{punto di accumulazione} di $A$ \\

\noindent
Notazione: \\
p.a. di $A$ = punto di accumulazione di $A$ \\

\noindent
Osservazioni: \\
La definizione di punto di accumulazione non richiede che $x_0 \in A$ \\
Ogni punto interno è anche un punto di accumulazione.


\subsection{Definizione di Punto Isolato}

\noindent
$A \subseteq \R,$ $x_0 \in \R$ \\

$\exists \varepsilon > 0 : I_\varepsilon(x_0) \cap A = \braces{x_0} \implies$ $x_0$ è un \textbf{punto isolato} di $A$


\subsection{Definizione di Punto Aderente}

\noindent
$x_0$ è un punto di accumulazione di $A$ $\lor$ $x_0$ è un punto isolato di $A \implies x_0$ è un \textbf{punto aderente} ad $A$


\subsection{Definizione di Parte Interna}

\noindent
$A \subseteq \R$ \\

$\mathring A = \braces{x \in A : x \text{ è un punto interno di } A}$


\subsection{Definizione di Chiusura}

\noindent
$A \subseteq \R$ \\

$\overbar A = \braces{x \in A : x \text{ aderente ad } A}$ \\

\noindent
N.B. \\
$\mathring A \subseteq A \subseteq \overbar A$


\subsection{Definizione di Frontiera}

\noindent
$A \subseteq \R$ \\

$\partial A = \overbar A \setminus \mathring A = \braces{x \in \overbar A : x \not\in \mathring A}$


\subsection{Definizione di Insieme Aperto}

\noindent
$A \subseteq \R$ \\

$A = \mathring{A} \implies A$ è \textbf{aperto} (contiene solo punti interni)


\subsection{Definizione di Insieme Chiuso}

$A = \overbar A \implies A$ è \textbf{chiuso}



\chapter{Limiti}



\section{Definizione di Limite}

\noindent
$A \subseteq \R,$ $x_0 \in \R$ punto di accumulazione di $A,$ $f: A \rightarrow \R$ \\

$f$ converge a $L \in \R$ per $x$ che tende ad $x_0$ scritto: \\

$\lim\limits_{x \to x_0} f(x) = L$ \\

\noindent
se: \\

$\forall \varepsilon > 0 \ldotp \exists \delta > 0 : \forall x \in I_\delta(x_0) \cap (A \setminus \braces{x_0} \ldotp \abs{f(x) - L} < \varepsilon$ \\

\noindent
N.d.r. \\
Tutti i $\delta$ nelle definizioni e dimostrazioni sono da considerarsi $\in \R$ salvo diversamente specificato. \\

\noindent
Osservazioni: \\
La definizione non richiede che $x_0 \in A$ \\
Anche se $x_0 \in \dom(f) = A$ il valore della funzione in questo punto non ha nessuna influenza sul valore del limite. \\
$x_0$ deve essere un p.a. di $A$ perché x deve potersi avvicinare a $x_0$ indefinitamente rimanendo in $A = \dom(f)$.



\section{Estensione della Definizione del Limite}

\noindent
Estensione della definizione di: \\

$\lim\limits_{x \to x_0} f(x) = L$ \\

\noindent
nei casi in cui $x_0 \in \braces{+\infty, -\infty}$ e/o $L \in \braces{+\infty, -\infty}$ \\

\noindent
N.d.r. \\
Tutte le definizioni possono essere riscritte equivalentemente sostituendo l'intorno di $\pm$ infinito $I_{\pm \infty}(a)$ dove compaiono gli intervalli $(a, +\infty)$ e $(-\infty, a)$.


\subsection{$x_0 \in \R,$ $L \in \braces{+\infty, -\infty}$}

\noindent
$f : A \rightarrow \R,$ $x_0$ p.a. di $A$


\subsubsection{$L = +\infty$}

\noindent
Si scrive $\lim\limits_{x \to x_0} = +\infty$ se: \\

\noindent
$\forall M \in \R \ldotp \exists \delta > 0 : \forall x \in I_\delta(x_0) \cap (A \setminus \braces{x_0}) \ldotp f(x) > M$ \\

\noindent
$f$ diverge positivamente per $x \to x_0$


\subsubsection{$L = -\infty$}

\noindent
Si scrive $\lim\limits_{x \to x_0} = -\infty$ se: \\

\noindent
$\forall M \in \R \ldotp \exists \delta > 0 : \forall x \in I_\delta(x_0) \cap (A \setminus \braces{x_0}) \ldotp f(x) < M$ \\

\noindent
$f$ diverge negativamente per $x \to x_0$


\subsection{$x_0 \in \braces{+\infty, -\infty},$ $L \in \R$}


\subsubsection{$f: [R, +\infty) \rightarrow \R,$ $R \in \R$}

\noindent
Si scrive $\lim\limits_{x \to x_0} = L$ se: \\

\noindent
$\forall \varepsilon > 0 \ldotp \exists a > R : \forall x \in (a, +\infty) \ldotp \abs{f(x) - L} < \varepsilon$


\subsubsection{$f: (-\infty, R] \rightarrow \R,$ $R \in \R$}

\noindent
Si scrive $\lim\limits_{x \to x_0} = L$ se: \\

\noindent
$\forall \varepsilon > 0 \ldotp \exists a < R : \forall x \in (-\infty, a) \ldotp \abs{f(x) - L} < \varepsilon$ \\

\noindent
$\abs{f(x) - L} < \varepsilon \iff f(x) \in (L - \varepsilon, L + \varepsilon) = I_\varepsilon(L)$


\subsection{$x_0 \in \braces{+\infty, -\infty},$ $L \in \braces{+\infty, -\infty}$}


\subsubsection{$x_0 = +\infty,$ $L = +\infty,$ $f : [R, +\infty) \rightarrow \R,$ $R \in \R$}

\noindent
Si scrive $\lim\limits_{x \to x_0} f(x) = +\infty$ se: \\

\noindent
$\forall M \in \R \ldotp \exists a > R : \forall x \in (a, +\infty) \ldotp f(x) > M$


\subsubsection{$x_0 = +\infty,$ $L = -\infty,$ $f : [R, +\infty) \rightarrow \R,$ $R \in \R$}

\noindent
Si scrive $\lim\limits_{x \to x_0} f(x) = -\infty$ se: \\

\noindent
$\forall M \in \R \ldotp \exists a > R : \forall x \in (a, +\infty) \ldotp f(x) < M$


\subsubsection{$x_0 = -\infty,$ $L = +\infty,$ $f : (-\infty, R] \rightarrow \R,$ $R \in \R$}

\noindent
Si scrive $\lim\limits_{x \to x_0} f(x) = +\infty$ se: \\

\noindent
$\forall M \in \R \ldotp \exists a < R : \forall x \in (-\infty, a) \ldotp f(x) > M$


\subsubsection{$x_0 = -\infty,$ $L = -\infty,$ $f : (-\infty, R] \rightarrow \R,$ $R \in \R$}

\noindent
Si scrive $\lim\limits_{x \to x_0} f(x) = -\infty$ se: \\

\noindent
$\forall M \in \R \ldotp \exists a < R : \forall x \in (-\infty, a) \ldotp f(x) < M$



\section{Definizione Disuguaglianza Triangolare}

\noindent
$\forall a,b \in \R \ldotp \abs{a + b} \leq \abs{a} + \abs{b}$ \\

\noindent
$\forall x \in \R \ldotp x \leq \abs{x} \land -x \leq \abs{x}$



\section{Teorema di Unicità del Limite}

\noindent
$f : A \rightarrow \R,$ $x_{0} \in \R$ p.a. di $A$ \\

\noindent
Supponendo che esistano due limiti $L \in \R$ e $L' \in \R$ tali che $\lim\limits_{x \to x_0} f(x) = L$ e contemporaneamente $\lim\limits_{x \to x_0} f(x) = L'$. \\

\noindent
Allora $L = L'$



\subsection{Dimostrazione}

\noindent
Sia $\varepsilon > 0$ arbitrario.

\noindent
Supponendo per assurdo che estano due limiti $L$ ed $L'$, con $L \neq L'$: \\

\noindent
$\lim\limits_{x \to x_0} f(x) = L \implies \exists \delta_1 > 0 : \forall x \in {I_\delta}_1(x_0) \cap (A \setminus \braces{x_0}) \ldotp \abs{f(x) - L} < \varepsilon$ \\

\noindent
$\lim\limits_{x \to x_0} f(x) = L' \implies \exists \delta_2 > 0 : \forall x \in {I_\delta}_2(x_0) \cap (A \setminus \braces{x_0}) \ldotp \abs{f(x) - L'} < \varepsilon$ \\\\

\noindent
$I_{\min(a, b)}(x_0) \subseteq I_{\max(a, b)}(x_0)$ \\

\noindent
Equivalentemente: \\

\noindent
$I_a(x_0) \cap I_b(x_0) = I_{\min(a, b)}(x_0)$ \\\\


\noindent
Ponendo $\delta = \min(\delta_1, \delta_2)$, nel risultante intorno: \\

$I_\delta(x_0) \cap (A \setminus \braces{x_0})$ \\

\noindent
valgono entrambe le definizioni dei limiti: \\

$\forall x \in I_\delta(x_0) \cap (A \setminus \braces{x_0}) \ldotp$ \\\\
\indent
$\abs{f(x) - L} < \varepsilon$ \\
\indent
$\abs{f(x) - L'} < \varepsilon$ \\

\noindent
La differenza assoluta è commutativa, quindi la prima si può riscrivere come: \\

$\forall x \in I_\delta(x_0) \cap (A \setminus \braces{x_0}) \ldotp$ \\\\
\indent
$\abs{L - f(x)} < \varepsilon$ \\

\noindent
e applicando la disuguaglianza triangolare si ottiene: \\

$\forall x \in I_\delta(x_0) \cap (A \setminus \braces{x_0}) \ldotp$ \\\\
\indent
$\abs{L - L'} = \abs{L - f(x) + f(x) - L'} \leq \abs{L - f(x)} + \abs{f(x) - L'}$ \\

\noindent
Viste le disuguaglianze: \\

$\forall x \in I_\delta(x_0) \cap (A \setminus \braces{x_0}) \ldotp$ \\\\
\indent
$\abs{f(x) - L} < \varepsilon$ \\
\indent
$\abs{f(x) - L'} < \varepsilon$ \\

\noindent
e dato che: \\

$a \geq 0,$ $b \geq 0,$ $c > 0$ \\
\indent
$a < c \land b < c \implies a + b < 2c$ \\

\noindent
è sicuramente vero che: \\

$\forall x \in I_\delta(x_0) \cap (A \setminus \braces{x_0}) \ldotp$ \\\\
\indent
$\abs{L - L'} \leq \abs{L - f(x)} + \abs{f(x) - L'} < 2 \varepsilon$ \\

\noindent
Dunque $\forall \varepsilon > 0$ si ha: \\

$0 \leq \abs{L - L'} < 2 \varepsilon \implies \abs{L - L'} = 0$ \\
\indent
$\abs{L - L'} = 0 \implies L = L'$ \\

C.V.D.



\section{Algebra dei Limiti}

\noindent
$f, g : A \rightarrow \R,$ $x_0 \in \overbar\R,$ $L, M \in \overbar\R$ tali che $x_0$ è un p.a. di $A$ e: \\

\noindent
$\lim\limits_{x \to x_0} f(x) = L$ \\
$\lim\limits_{x \to x_0} g(x) = M$ \\

\noindent
Allora le seguenti identità: \\
$\lim\limits_{x \to x_0} (f(x) + g(x)) = L + M$ \\
$\lim\limits_{x \to x_0} (f(x) g(x)) = L \cdot M$ \\
$\lim\limits_{x \to x_0} (\frac{f(x)}{g(x)}) = \frac{L}{M}$ \\

\noindent
valgono in assenza di forme indeterminate ($\infty - \infty, 0 \cdot (\pm \infty), \frac{\pm \infty}{\pm \infty}$)


\subsection{Caso Particolare di $\lim\limits_{x \to x_0} (\frac{f(x)}{g(x)}) = \frac{L}{M}$}

\noindent
Supponendo $\lim\limits_{x \to x_0} f(x) = L \neq 0 \land \lim\limits_{x \to x_0} g(x) = 0$ \\

\noindent
Allora valgono le seguenti regole: \\

\noindent
Se $\exists \delta > 0 : \forall x \in I_\delta(x_0) \cap (A \setminus \braces{x_0}) \ldotp g(x) > 0$ allora: \\

\noindent
$\lim\limits_{x \to x_0} (\frac{f(x)}{g(x)}) =$ \\

\noindent
$+\infty$ se $L > 0$ \\
$-\infty$ se $L < 0$ \\

\noindent
Se $\exists \delta > 0 : \forall x \in I_\delta(x_0) \cap (A \setminus \braces{x_0}) \ldotp g(x) < 0$ allora: \\

\noindent
$\lim\limits_{x \to x_0} (\frac{f(x)}{g(x)}) =$ \\

\noindent
$-\infty$ se $L > 0$ \\
$+\infty$ se $L < 0$ \\

\noindent
Se la funzione cambia segno in \underline{ogni} intorno di $x_0$, ovvero: \\
$\forall \delta > 0 \ldotp \exists x_1, x_2 \in I_\delta(x_0) \cap (A \setminus \braces{x_0}) : g(x_1)g(x_2) < 0$ allora: \\

\noindent
$\lim\limits_{x \to x_0} (\frac{f(x)}{g(x)})$ non esiste



\section{Teorema della Permanenza del Segno}

\noindent
$A \subseteq \R,$ $x_0 \in \R$ p.a. di $A,$ $f : A \rightarrow \R$ \\

\noindent
Suppongo che: \\

\noindent
$\lim\limits_{x \to x_0} f(x) = L \neq 0$ \\

\noindent
allora $\exists \delta > 0 : \forall x \in I_\delta(x_0) \cap (A \setminus \braces{x_0}) \ldotp f(x)$ ha lo stesso segno di $L$.


\subsection{Dimostrazione}

\noindent
Supponendo che $L > 0$ si pone nella definizione di $\lim\limits_{x \to x_0} = L$ $\varepsilon = \frac{L}{2} > 0$. \\

\noindent
Quindi $\exists \delta > 0 : \forall x \in I_\delta(x_0) \cap (A \setminus \braces{x_0}) \ldotp \abs{f(x) - L} < \frac{L}{2} \iff f(x) \in (\frac{L}{2}, \frac{3L}{2}) \implies f(x) > 0$ \\

\noindent Q.E.D.


\section{Teorema del confronto}

\noindent
$f, g : A \rightarrow \R,$ $x_0$ p.a. di $A$. \\

\noindent
Supponendo che: \\

$\forall x \in A \ldotp f(x) \leq g(x)$ \\

\noindent
Se i limiti: \\

$\lim\limits_{x \to x_0} f(x) = L$ \\
\indent
$\lim\limits_{x \to x_0} g(x) = M$ \\

\noindent
esistono e sono finiti, allora $L \leq M$


\subsection{Dimostrazione}

\noindent
Supponendo per assurdo che $L > M$ si considera la funzione $f(x) - g(x)$ in $A$ e si osserva che: \\

$\lim\limits_{x \to x_0} (f(x) - g(x)) = L - M > 0$ \\

\noindent
Quindi per il teorema della permanenza del segno: \\

$\exists \delta > 0 : \forall x \in I_\delta(x_0) \cap (A \setminus \braces{x_0}) \ldotp f(x) - g(x) > 0$ \\

\noindent
Contraddizione con l'ipotesi! Q.E.D.


\subsubsection{Attenzione}

\noindent
Da $f(x) < g(x)$ non segue che: \\

$\lim\limits_{x \to x_0} f(x) < \lim\limits_{x \to x_0} g(x)$



\section{Teorema dei Due Carabinieri}

\noindent
$f, g, h : A \rightarrow \R,$ $x_0$ p.a. di $A$. \\

\noindent
Supponendo che: \\

$\forall x \in A \ldotp h(x) \leq f(x) \leq g(x)$ \\

\noindent
$\lim\limits_{x \to x_0} h(x) = \lim\limits_{x \to x_0} g(x) = L \implies \lim\limits_{x \to x_0} f(x) = L$


\subsection{Dimostrazione}

\noindent
$\varepsilon$ arbitrario. \\

$\lim\limits_{x \to x_0} h(x) = L \implies \exists \delta_1 > 0 : \forall x \in I_{\delta_1}(x_0) \cap (A \setminus \braces{x_0}) \ldotp \abs{h(x) - L} < \varepsilon$

$\lim\limits_{x \to x_0} g(x) = L \implies \exists \delta_2 > 0 : \forall x \in I_{\delta_2}(x_0) \cap (A \setminus \braces{x_0}) \ldotp \abs{g(x) - L} < \varepsilon$ \\

\noindent
Ponendo $\delta = \min(\delta_1, \delta_2)$ si ha che: \\

$\forall x \in I_{\delta}(x_0) \cap (A \setminus \braces{x_0})$ \\

\noindent
valgono: \\

$f(x) - L \leq g(x) - L < \varepsilon$

$f(x) - L \geq h(x) - L \geq -\abs{h(x) - L} > -\varepsilon$ \\

\noindent
Che implicano: \\

$-\varepsilon < f(x) - L < \varepsilon \iff \abs{f(x) - L} < \varepsilon$ \\

\noindent
Q.E.D.


\subsection{Corollario}

\noindent
$f, g : A \rightarrow \R,$ $x_0$ p.a. di $A$. \\

\noindent
Supponendo che: \\

$\lim\limits_{x \to x_0} f(x) = 0$ \\

la funzione $g$ è limitata in $A$, cioè:

$\exists M \in R : \forall x \in A \ldotp \abs{g(x)} \leq M$ \\

\noindent
Allora: \\

$\lim\limits_{x \to x_0} f(x) g(x) = 0$


\subsubsection{Dimostrazione}

\noindent
Siccome $\forall x \in A \ldotp \abs{g(x)} \leq M$ si ha che: \\

$\forall x \in A \ldotp 0 \leq \abs{f(x)g(x)} \leq M \abs{f(x)}$ \\

\noindent
Applico il teorema dei due carabinieri con $h(x) = 0$ e $g(x) = M \abs{f(x)}$. \\

\noindent
Si ha $\lim\limits_{x \to x_0} M \abs{f(x)} = M \lim\limits_{x \to x_0} \abs{f(x)} = 0$ \\

\noindent
$\lim\limits_{x \to x_0} \abs{f(x)} = 0$ \\

\noindent
Quindi: \\

$\lim\limits_{x \to x_0} \abs{f(x)g(x)} = 0$ da cui la tesi. \\

\noindent
Q.E.D



\section{Limiti Unilaterali}


\subsection{Limite Destro}

\noindent
$f : A \rightarrow \R,$ $x_0 \in \R$ p.a. di $A \cap (x_0, +\infty)$ \\

\noindent
$f$ ammette \textbf{limite destro} in $x_0$, scritto: \\

$\lim\limits_{x \to x_0^+} f(x) = L \in \R$ \\

\noindent
se: \\

$\forall \varepsilon > 0 \ldotp \exists \delta > 0 : \forall x \in (x_0, x_0 + \delta) \cap A \ldotp \abs{f(x) - L} < \varepsilon$ \\

\noindent
N.B. \\

\noindent
Se $A = (a, b)$ allora $b$ è un p.a. di $A$, ma non è un p.a. di $A \cap (b, +\infty) = \emptyset$


\subsection{Limite Sinistro}

\noindent
$f : A \rightarrow \R,$ $x_0 \in \R$ p.a. di $A \cap (-\infty, x_0)$ \\

\noindent
$f$ ammette \textbf{limite sinistro} in $x_0$, scritto: \\

$\lim\limits_{x \to x_0^-} f(x) = L \in \R$ \\

\noindent
se: \\

$\forall \varepsilon > 0 \ldotp \exists \delta > 0 : \forall x \in (x_0 - \delta, x_0) \cap A \ldotp \abs{f(x) - L} < \varepsilon$


\subsection{Caso $L \in \braces{+\infty, -\infty}$}


\subsubsection{Limite Destro}

\noindent
$f : A \rightarrow \R,$ $x_0 \in \R$ p.a. di $A \cap (x_0, +\infty)$ \\

\noindent
Si scrive: \\

$\lim\limits_{x \to x_0^+} f(x) = +\infty$ \\

\noindent
se: \\

$\forall M \in \R \ldotp \exists \delta > 0 : \forall x \in (x_0, x_0 + \delta) \cap A \ldotp f(x) > M$ \\

\noindent
Analogamente, si scrive: \\

$\lim\limits_{x \to x_0^+} f(x) = -\infty$ \\

\noindent
se: \\

$\forall m \in \R \ldotp \exists \delta > 0 : \forall x \in (x_0, x_0 + \delta) \cap A \ldotp f(x) < m$


\subsection{Limite Sinistro}

\noindent
$f : A \rightarrow \R,$ $x_0 \in \R$ p.a. di $A \cap (-\infty, x_0)$ \\

\noindent
Si scrive: \\

$\lim\limits_{x \to x_0^-} f(x) = +\infty$ \\

\noindent
se: \\

$\forall M \in \R \ldotp \exists \delta > 0 : \forall x \in (x_0 - \delta, x_0) \cap A \ldotp f(x) > M$ \\

\noindent
Analogamente si scrive: \\

$\lim\limits_{x \to x_0^-} f(x) = -\infty$ \\

\noindent
se: \\

$\forall m \in \R \ldotp \exists \delta > 0 : \forall x \in (x_0 - \delta, x_0) \cap A \ldotp f(x) < m$


\subsection{Legame con il Limite}

\noindent
$f : A \rightarrow \R,$ $x_0 \in \R$ p.a. di $A \cap (x_0, +\infty)$ e di $A \cap (-\infty, x_0)$ \\

\noindent
Allora $f$ ammette limite per $x \to x_0$ se e solo se: \\

$\exists \lim\limits_{x \to x_0^+} f(x) = L^+ \in \overbar\R$ \\

$\exists \lim\limits_{x \to x_0^-} f(x) = L^- \in \overbar\R$ \\

$L^+ = L^-$ \\

\noindent
In tal caso: \\

$\lim\limits_{x \to x_0} f(x) = L^+ = L^-$ \\

\noindent
N.B. \\
Il teorema implica che: \\

$\lim\limits_{x \to x_0} f(x)$ \textbf{non esiste} se: \\

\noindent
Almeno uno fra: \\

$\lim\limits_{x \to x_0^+} f(x)$ \\

$\lim\limits_{x \to x_0^-} f(x)$ \\

\noindent
non esiste, oppure se: \\

$\lim\limits_{x \to x_0^+} f(x) \neq \lim\limits_{x \to x_0^-} f(x)$



\section{Funzioni Monotone}

\noindent
Si dice che $f : A \rightarrow \R$ è: \\

\noindent
\textbf{Crescente} in $A$ se: \\

$\forall x, y \in A \ldotp x < y \implies f(x) \leq f(y)$ \\

\noindent
\textbf{Decrescente} in $A$ se: \\

$\forall x, y \in A \ldotp x < y \implies f(x) \geq f(y)$ \\

\noindent
\textbf{Strettamente Crescente} in $A$ se: \\

$\forall x, y \in A \ldotp x < y \implies f(x) < f(y)$ \\

\noindent
\textbf{Strettamente Decrescente} in $A$ se: \\

$\forall x, y \in A \ldotp x < y \implies f(x) > f(y)$ \\

\noindent
N.B. \\
$f$ è strettamente monotona $\implies f$ è iniettiva \\
$f$ è crescente e decrescente $\implies f$ è costante


\subsection{Teorema}

\noindent
$f : A \rightarrow \R,$ $x_0 \in \R$ p.a. di $A \cap (x_0, +\infty)$ e di $A \cap (-\infty, x_0)$ \\

\noindent
$f$ è crescente in $A \implies f$ ammette in $x_0$ entrambi i limiti unilaterali e vale: \\

$\lim\limits_{x \to x_0^+} f(x) = \inf \braces{f(x) : x > x_0, x \in A}$ \\

$\lim\limits_{x \to x_0^-} f(x) = \sup \braces{f(x) : x < x_0, x \in A}$


\subsubsection{Dimostrazione}

\noindent
$l = \inf \braces{f(x) : x > x_0, x \in A}$ \\

\noindent
Quindi: \\

\noindent
$\forall x > x_0, x \in A \ldotp l \leq f(x)$ \\

\noindent
$\forall \varepsilon > 0 \ldotp \exists x_\varepsilon > x_0, x_\varepsilon \in A : f(x_\varepsilon) < l + \varepsilon$ \\

\noindent
$\varepsilon > 0$, ponendo $\delta = x_\varepsilon - x_0 > 0$ allora: \\

\noindent
$\forall x \in (x_0, x_0 + \delta) \cap A = (x_0, x_\varepsilon) \cap A \ldotp$ \\

$0 \leq f(x) - l \leq f(x_\varepsilon) - l < \varepsilon$ \\

\noindent
$f$ è crescente, quindi: \\

$\abs{f(x) - l} < \varepsilon$ \\

$\lim\limits_{x \to x_0^+} f(x) = l$ \\

\noindent
$L = \sup \braces{f(x) : x < x_0; x \in A}$ \\

\noindent
Quindi: \\

$\forall x < x_0, x \in A \ldotp f(x) \leq L$ \\

$\forall \varepsilon > 0 \ldotp \exists x_\varepsilon < x_0, x_\varepsilon \in A : f(x_\varepsilon) > L - \varepsilon$ \\

\noindent
$\varepsilon > 0$, ponendo $\delta = x_0 - x_\varepsilon > 0$ allora: \\

\noindent
$\forall x \in (x_0 - \delta, x_0) \cap A = (x_\varepsilon, x_0) \cap A \ldotp$ \\

$0 \leq L - f(x) \leq L - f(x_\varepsilon) < \varepsilon$ \\

\noindent
$f$ è crescente, quindi: \\

$x_\varepsilon < x \implies f(x_\varepsilon) < f(x)$ \\

$\abs{f(x) - L} < \varepsilon$ \\

$\lim\limits_{x \to x_0^-} f(x) = L$ \\

\noindent
Q.E.D. \\

\noindent
N.B. \\
$L$ e $l$ possono essere diversi.


\subsection{Teorema}

\noindent
$f : A \rightarrow \R,$ $x_0 \in \R$ p.a. di $A \cap (x_0, +\infty)$ e di $A \cap (-\infty, x_0)$ \\

\noindent
$f$ è decrescente in $A \implies f$ ammette in $x_0$ entrambi i limiti unilaterali e vale: \\

$\lim\limits_{x \to x_0^+} f(x) = \sup \braces{f(x) : x > x_0, x \in A}$ \\

$\lim\limits_{x \to x_0^-} f(x) = \inf \braces{f(x) : x < x_0, x \in A}$ \\

\noindent
``La monotonia è madre dei limiti unilaterali''



\section{Limiti Notevoli}

\noindent
$\forall a \in \R \ldotp \lim\limits_{x \to 0} \frac{\sin(ax)}{x} = a$ \\

\noindent
$\forall a \in \R \ldotp \lim\limits_{x \to 0} \frac{e^{ax} - 1}{x} = a$ \\

\noindent
$\forall a \in \R \ldotp \lim\limits_{x \to 0} \frac{\log(1+ax)}{x} = a$ \\

\noindent
$\forall a \in \R \ldotp \lim\limits_{x \to 0} \frac{(1+x)^a-1}{x} = a$ \\

\noindent
$\forall a \in \R \ldotp \lim\limits_{x \to +\infty} (1 + \frac a x)^x = e^a$ \\

\noindent
$e$ = costante di Eulero.



\section{Limiti di Funzioni Composte}

\noindent
$f : A \rightarrow B \subseteq \R,$ $g : B \rightarrow \R$ \\

\noindent
Se: \\

$\lim\limits_{x \to x_0} f(x) = y_0$ \\

$\lim\limits_{y \to y_0} g(y) = L$ \\

$\exists \delta > 0 : 0 < \abs{x - x_0} < \delta \implies f(x) \neq y_0$ \\

\noindent
Allora:

\noindent
$\lim\limits_{x \to x_0} g(f(x)) = L$



\chapter{Funzioni Continue}



\section{Definizione di Continuità}

\noindent
Si dice che $f : A \subseteq \R \rightarrow \R$ è \textbf{continua} in $x_0 \in A$ se: \\

$\forall \varepsilon > 0 \ldotp \exists \delta > 0 : \forall x \in I_\delta(x_0) \cap A \ldotp \abs{f(x) - f(x_0)} < \varepsilon$ \\

\noindent
Si dice che $f$ è continua in $E \subseteq A$ se è continua $\forall x \in E$ \\\\



\section{Teorema (Limiti e Continuità)}

\noindent
$f : A \subseteq \R \rightarrow \R$, $x_0$ p.a. di $A$ \\

\noindent
$f$ è continua in $x_0 \iff \lim\limits_{x \to x_0} f(x) = f(x_0)$



\section{Teorema (Continuità della Somma e del Prodotto)}

% TODO



\section{Teorema}

% TODO



\section{Teorema degli Zeri di Bolzano}

% TODO



\section{Min, max, inf, sup di una Funzione}

% TODO



\section{Teorema di Weierstrass}

% TODO



\section{Teorema dei Valori Intermedi}

% TODO



\section{Funzioni Continue Invertibili}

% TODO



\section{Caratterizzazioni di Funzioni Continue Invertibili}

% TODO



\section{Punti di Discontinuità}

% TODO



\chapter{Successioni}



\section{Definizione}

\noindent
Una \textbf{successione numerica} è una funzione $f$ definita in $dom(f) \subseteq \N$ che ad ogni $n \in dom(f)$ associa un numero reale \\

$a_n = f(n)$ \\

\noindent
Notazione: \\

Se: $dom(f) = \braces{n \in \N : n \geq n_0, n_0 \in \N}$ \\

allora si scrive: $\braces{a_n}_{n \geq n_0}$ \\

se: $dom(f) = \N$ \\

allora si scrive: $\braces{a_n}_{n \in \N}$ \\

% TODO +\infty unico p.a. di \N



\section{Limiti di Successioni Numeriche}

% TODO



\section{Algebra dei Limiti}

% TODO



\section{Teorema della Permanenza del Segno}

% TODO



\section{Teorema del Confronto}

% TODO



\section{Teorema dei Due Carabinieri}

% TODO

\end{document}

% Fonti:

% - Miei appunti Hynek Kovarik
% - https://hynek-kovarik.unibs.it/courses/analisi-1-22.htm
% - Appunti Davide
% - Appunti Mary
% - https://it.wikipedia.org/wiki/Funzione_(matematica)
% - https://it.wikipedia.org/wiki/Funzione_iniettiva
% - https://it.wikipedia.org/wiki/Funzione_suriettiva
% - https://it.wikipedia.org/wiki/Corrispondenza_biunivoca
% - https://en.wikipedia.org/wiki/Restriction_(mathematics)
