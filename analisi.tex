\documentclass[a4paper, twoside, italian, 11pt]{book}

\usepackage{amsfonts}  % for mathbb
\usepackage{amsmath}   % for implies
\usepackage{indentfirst}
\usepackage{amssymb}   % for emptyset

\newcommand{\braces}[1] {\left\{#1\right\}}
\newcommand{\overbar}[1]{\mkern 1.5mu\overline{\mkern-1.5mu#1\mkern-1.5mu}\mkern 1.5mu}

\newcommand{\N}{\mathbb{N}}
\newcommand{\Z}{\mathbb{Z}}
\newcommand{\Q}{\mathbb{Q}}
\newcommand{\K}{\mathbb{K}}
\newcommand{\R}{\mathbb{R}}
\newcommand{\E}{\mathbb{E}}

\let\emptyset\varnothing

\begin{document}

\pagestyle{headings}

\frontmatter

\begin{titlepage}
	\begin{center}
		{\huge \bfseries Appunti di Analisi Matematica I\\}
		\vspace{1.5cm}
		{\Large \bfseries Ettore Forigo}
	\end{center}
\end{titlepage}

\mainmatter

\chapter{1}



\section{Definizione del Principio di Induzione}

\indent
$(P(n_0) \land P(n) \implies P(n + 1)) \implies \forall n \in \N \ldotp P(n)$ \\

($\implies$ ha precedenza su $\land$) \\

\noindent
Il caso base nell'induzione può essere anche un numero $\neq 0$.




\section{Insiemi Famosi}

\noindent
$\N$ = Numeri Naturali $= \braces{0, 1, 2, 3, ...}$ \\
$\Z$ = Numeri Interi \\
$\Q$ = Numeri Razionali \\

\noindent
$\N \subset \Z \subset \Q$ \\

\noindent
Su $\Q$ è definita una relazione d'ordine totale  ($\leq$) \\

\noindent
Gli insiemi con relazioni d'ordine totale si chiamano totalmente ordinati.


\section{Campo Ordinato dei Razionali}
\noindent
$(\Q, \leq)$ formano un Campo Ordinato.


\section{Definizione di Completezza di un Campo}

\noindent
Un campo totalmente ordinato $(\K, \leq)$ si dice completo se vale il seguente \textbf{assioma di completezza} (Assioma di Dedekin): \\

$\forall A, B, A \subseteq \K, B \subseteq K, A \neq \emptyset, B \neq \emptyset$ \\

$\forall x \in A, \forall y \in B \ldotp x \leq y \implies \exists c \in \K : \forall x \in A, \forall y \in B \ldotp x \leq c \leq y$ \\

\noindent
Chiamiamo $c$ elemento separatore tra gli insiemi A e B. \\

\noindent
Il campo $(\Q, \leq)$ è totalmente ordinato ma non completo.


\section{Definizione dei Numeri Reali}

\noindent
$\R$ è una estensione di $\Q$ tale che il campo $(\R, \leq)$ è totalmente ordinato e completo.


\subsection{Interpretazione Geometrica}

\noindent
Ogni numero reale può essere univocamente associato ad un punto della retta reale e viceversa



\section{Definizione Numeri Irrazionali}

\noindent
$\R \setminus \Q$ = Numeri Irrazionali



\section{Definizione di Massimi e Minimi}

$\E \subseteq \R, \E \neq \emptyset$ \\

$\exists a \in \E : \forall x \in \E \ldotp a \leq x \implies a$ è un minimo di $\E$ \\

$\exists b \in \E : \forall x \in \E \ldotp x \leq b \implies b$ è un massimo di $\E$ \\

$min(\E) = a$ \\
\indent
$max(\E) = b$ \\

\noindent
Esistono insiemi limitati che non ammettono né massimo né minimo. \\

$\E = \braces{x \in \R : 0 < x < 1}$


\subsection{Lemma: Unicità di min e max}

Se $\E \subseteq \R$ ammette minimo o massimo, allora è unico.

% TODO: Dimostrazione



\section{Definizione di Maggioranti e Minoranti}

$\E \subseteq \R, \E \neq \emptyset$ \\

$a \in \R$ è un maggiorante di $\E$ se $\forall x \in \E \ldotp a \leq x$ \\

$b \in \R$ è un maggiorante di $\E$ se $\forall x \in \E \ldotp x \leq b$ \\

\noindent
\textbf{Non sono unici!} \\

\noindent
$M(\E)$ = Insieme dei maggioranti di $\E$\\
$m(\E)$ = Insieme dei minoranti di $\E$



\section{Definizione di Insieme Limitato}

\noindent
$E \subseteq \R, \E \neq \emptyset$ \\

\noindent
$M(\E) \neq \emptyset \implies \E$ è superiormente limitato \\
$m(\E) \neq \emptyset \implies \E$ è inferiormente limitato \\
$M(\E) \neq \emptyset \land m(\E) \neq \emptyset \implies \E$ è limitato



\section{Teorema}

\noindent
$\E \subseteq \R, \E \neq \emptyset$ \\

\noindent
$\E$ è superiormente limitato $\implies M(\E)$ ammette minimo (estremo superiore di $\E$) \\

\noindent
$\E$ è inferiormente limitato $\implies m(\E)$ ammette massimo (estremo inferiore di $\E$)

% TODO: Dimostrazione



\section{Definizione di Estremo Superiore ed Inferiore}

\noindent
$\E$ è superiormente limitato $\implies sup(\E) = sup \E = min(M(\E))$ \\

\noindent
$\E$ è inferiormente limitato $\implies inf(\E) = inf \E = max(m(\E))$


\subsection{Proprietà}

\noindent
$sup$ $\E \in \E \implies sup$ $\E = max$ $\E$ \\
$inf$ $\E \in \E \implies inf$ $\E = min$ $\E$ \\
$sup$ $\E$ e $inf$ $\E$ sono unici.



\section{Caratterizzazione di sup e inf}

\noindent
$\E \subseteq \R,$ $\E \neq \emptyset,$ $\E$ superiormente limitato


\subsection{Caratterizzazione di sup}

\noindent
$\iota = sup$ $\E \iff \forall x \in \E : x \leq \iota$ $\land$ $\forall \varepsilon > 0$ $\exists x \in \E : x > \iota - \varepsilon$


\subsection{Caratterizzazione di inf}

\noindent
$\iota = inf$ $\E \iff \forall x \in \E : \iota \leq x$ $\land$ $\forall \varepsilon > 0$ $\exists x \in \E : x < \iota + \varepsilon$



\section{Definizione di $\overbar \R$}

\noindent
Insieme dei numeri reali estesi: \\
$\overbar \R = \R$ $\cup$ $\braces{-\infty, +\infty}$


\subsection {Relazione d'ordine $\leq$ e le operazioni somma e prodotto su $\overbar \R$}


\subsubsection{Relazione $\leq$}

\noindent
$\forall x \in \overbar \R: -\infty \leq x \leq +\infty$ \\
$\forall x \in \R: -\infty < x < +\infty$


\subsection{Somma}

\noindent
$\forall x \in \R: x + \infty = +\infty$ \\
$\forall x \in \R: x + (-\infty) = -\infty$


\subsection{Prodotto}

\noindent
$\forall x > 0, x \in \R$ \\
$x \cdot (+\infty) = +\infty$ \\
$x \cdot (-\infty) = -\infty$ \\


\noindent
$\forall x < 0,$ $x \in \R$ \\
$x \cdot (+\infty) = -\infty$ \\
$x \cdot (-\infty) = +\infty$ \\

\noindent
N.B. \\
Non sono definite le operazioni: \\
$0 \cdot (\pm \infty),$ $+ \infty - \infty$



\section{Intervalli}

\noindent
$I \subseteq \overbar \R : \forall x, y \in I : x < z < y \implies z \in I$ \\
$I$ è un detto intervallo. \\

\noindent
$a, b \in \overbar \R, a < b$ \\


\subsection{Intervallo aperto di estremi $a$ e $b$}

\noindent
$(a, b) = ]a, b[ = \braces{ x \in \overbar \R a < x < b}$ \\


\subsection{Intervallo semi-aperto a destra di estremi $a$ e $b$}

\noindent
$[a, b) = \braces{ x \in \overbar \R a \leq x < b}$ \\


\subsection{Intervallo semi-aperto a sinistra di estremi $a$ e $b$}

\noindent
$(a, b] = \braces{ x \in \overbar \R a < x \leq b}$ \\


\subsection{Intervallo chiuso di estremi $a$ e $b$}

\noindent
$[a, b] = \braces{ x \in \overbar \R a \leq x \leq b}$ \\


\subsection{}

\noindent
$\E \subseteq \R,$ $\E \neq \emptyset,$ $M(\E) = \emptyset$ \\
$sup$ $\E = +\infty$ \\

\noindent
$\E \subseteq \R,$ $\E \neq \emptyset,$ $m(\E) = \emptyset$ \\
$inf$ $\E = -\infty$



\section{Funzioni}

\noindent
Una funzione è definita da una terna $(f, A, B)$ dove: \\

\noindent
$A \subseteq \overbar \R,$ $B \subseteq \overbar \R,$ $A \neq \emptyset,$ $B \neq \emptyset$ \\
$f$ è una legge che ad ogni elemento $x \in A$ associa univocamente un elemento $f(x) \in B$. \\

\noindent
Notazione: \\
$A = dom(f)$ (dominio di $f$) \\
$B = codom(f)$ (codominio di $f$) \\

\noindent
Si scrive: $f : A \rightarrow B$ \\

\noindent
N.B. \\
Il codominio $B$ non è determinato univocamente da $f$. \\
Se $B$ è codominio di $f$ e $B \subseteq C$ allora anche $C$ è codominio di $f$. \\

\noindent
Due funzioni $f_1 : A_1 \rightarrow \R$ e $f_2 : A_2 \rightarrow \R$ \\
sono uguali $\iff A_1 = A_2$ $\land$ $\forall x \in A_1 = A_2 : f_1(x) = f_2(x)$


\subsection{Definizione di Insieme Immagine}

\noindent
$f : A \rightarrow B$ \\
$im(f) = f[A] =$ Im$f= \braces{y \in B : \exists x \in A : y = f(x)}$ \\

\noindent
$im(f) \subseteq codom(f)$


\subsection{Definizione di Iniettività}

\noindent
Una funzione da $A$ a $B$ si dice \textbf{iniettiva} se: \\

$\forall x, x' \in A \ldotp f(x) = f(x') \implies x = x'$


\subsection{Definizione di Suriettività}

\noindent
$im(f) = codom(f)$


\subsubsection{Interpretazione Geometrica}

\noindent
$\forall y_0 \in codom(f)$ la retta $y = y_0$ interseca il grafico di $f$ in almeno un punto. \\

\noindent
Equivalentemente: \\
\indent
$\forall y \in codom(f)$ \\
\indent
$f^{-1}({y}) \neq \emptyset$ \\

\noindent
Se $f : A \rightarrow B$ non è suriettiva si può rendere suriettiva restringendo il suo codominio alla sua immagine (Troncatura). \\

\noindent
Si può restringere anche il dominio per rendere la funzione iniettiva (Restrizione). % TODO Duplicate? Line 411


\subsection{Definizione di Biiettività}

Una funzione si dice \textbf{biiettiva} (o biiezione, o anche corrispondenza 1 a 1 o biunivoca) se è sia iniettiva che suriettiva.



\section{Definizione di Invertibilità}

\noindent
$\forall y \in B$ $\exists! x \in A : y = f(x) \implies f : A \rightarrow B$ è invertibile. \\

\noindent
$f : A \rightarrow B$ è invertibile $\implies f^{-1} : im(f) \rightarrow dom(f)$ è la funzione inversa di f. \\

\noindent
$\forall y \in (B = im(f)) : y = f(x) \iff x = f^{-1}(y)$ \\

\noindent
Osservazione: \\

\noindent
$\forall y \in im(f) : y = f(f^{-1}(y))$ \\

\noindent
$f$ è invertibile $\iff f$ è biiettiva \\

\noindent
Il grafico della funzione inversa: \\
$graf(f^{-1})$ \\
$= \braces{(y, x) \in B \times A : x = f^{-1}(y)}$ \\
$= \braces{(y, x) \in B \times A : y = f(x)}$ \\
$= \braces{(y, x) \in B \times A : (x, y) \in graf(f)}$ \\

\noindent
$(y, x) \in graf(f^{-1}) \iff (x, y) \in graf(f)$ \\

\noindent
$graf(f^{-1})$ è simmetrico di $graf(f)$ rispetto alla retta $y = x$



\section{Definizione di Restrizione}

\noindent
$f : A \rightarrow B,$ $E \subseteq A$ \\

\noindent
$f|_E : E \rightarrow B$ \\
$f|_E(x) = f(x)$ $\forall x \in E$ \\

\noindent
$f|_E$ è chiamata restrizione di $f$ ad $E$. \\

\noindent
Una funzione non iniettiva si può rendere iniettiva considerandone opportune restrizioni. % TODO Duplicate? Line 367

\end{document}

% Fonti:

% - Miei appunti Hynek Kovarik
% - Appunti Davide
% - https://it.wikipedia.org/wiki/Funzione_(matematica)
% - https://it.wikipedia.org/wiki/Funzione_iniettiva
% - https://it.wikipedia.org/wiki/Funzione_suriettiva
% - https://it.wikipedia.org/wiki/Corrispondenza_biunivoca
% - https://en.wikipedia.org/wiki/Restriction_(mathematics)
