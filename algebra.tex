\documentclass[a4paper, twoside, italian, 11pt]{book}

\usepackage{amsfonts}  % for mathbb
\usepackage{amsmath}   % for implies
\usepackage{centernot}
\usepackage{amssymb}   % for nexists

\newcommand{\braces}[1] {\left\{#1\right\}}
\newcommand{\N}{\mathbb{N}}
\newcommand{\Q}{\mathbb{Q}}
\newcommand{\Z}{\mathbb{Z}}
\newcommand{\R}{\mathbb{R}}
\newcommand{\C}{\mathbb{C}}
\newcommand{\K}{\mathbb{K}}

\begin{document}

\pagestyle{headings}

\frontmatter

\begin{titlepage}
	\begin{center}
		{\huge \bfseries Appunti di Algebra e Geometria\\}
		\vspace{1.5cm}
		{\Large \bfseries Ettore Forigo}
	\end{center}
\end{titlepage}

\mainmatter


\chapter{Insiemi}



\section{Cardinalità di Insiemi Finiti}

$\left | A \right | = n \in \N$ dove $n$ è il n° di elementi in $A$.



\section{Multiinsiemi o Sistemi}

Insieme con molteplicità, ovvero una collezione non ordinata di elementi con ripetizioni. \\

$[a, b, c]$



\section{Insiemi Famosi}

$\N =$ Numeri Naturali $= \braces{0, 1, 2, 3, ...}$ \\
$\Z =$ Numeri Interi \\
$\Q =$ Numeri Razionali \\
$\R =$ Numeri Reali \\
$\C =$ Numeri Complessi \\
$\N_0 = \N \setminus \braces{0}$ \\
$\Q^x = \Q \setminus \braces{0}$ \\
$\R^x = \R \setminus \braces{0}$ \\
$\C^x = \C \setminus \braces{0}$



\section{Definizione di Sequenza (o Ennupla / n-upla)}

Collezione ordinata di elementi. \\

$(a, b, c)$ \\

\noindent
$(a, b) := \braces{\braces{a}, \braces{a, b}}$ \\
$(a, a) = \braces{\braces{a}, \braces{a, a}} = \braces{\braces{a}, \braces{a}} = \braces{\braces{a}}$



\chapter{Relazioni}



\section{Definizione di Relazione}

Si dice che $R \subseteq A \times B$ è una relazione tra due insiemi $A$ e $B$.\\

\noindent
Se: \\
$C = \braces{(a_1, b_1), (a_2, b_2)}$ \\\\
Si scrive:\\
$b_1 = C(a_1)$ \\
$b_2 = C(a_2)$


\subsection{Proprietà delle Relazioni}


\subsubsection{Totalità a Sinistra}

Una relazione $R$ tra $A$ e $B$ si dice \textbf{ovunque definita} (o totale a sinistra) se: \\

$\forall x \in A \ldotp \exists y \in B : (x, y) \in R$


\subsubsection{Funzionalità}

Una relazione $R$ tra $A$ e $B$ si dice \textbf{funzionale} se: \\

$\forall x \in A \ldotp \exists y \in B : (x, y) \in R \implies \exists! y \in B : (x, y) \in R$



\section{Definizione di Funzione}

Una relazione $f$ si dice funzione se è funzionale e ovunque definita. \\

\noindent
``Funzione'' si riferisce alla terna: associazione di elementi, dominio e codominio, non solo all'associazione di elementi. Specificare solo un'associazione non definisce una funzione: occorre specificare anche dominio e codominio. Infatti, due funzioni che hanno una ``stessa'' associazione di elementi ma diverso dominio e/o diverso codominio sono funzioni diverse. \\

\noindent
Si scrive: \\
$f : A \rightarrow B$ \\
dove $A$ è il dominio di $f$ e $B$ è il codominio di $f$.


\subsection{Proprietà delle Funzioni}


\subsubsection{Iniettività}

Una funzione da $A$ a $B$ si dice \textbf{iniettiva} se: \\

$\forall x, x' \in A \ldotp f(x) = f(x') \implies x = x'$


\subsubsection{Suriettività}

Una funzione da $A$ a $B$ si dice \textbf{suriettiva} (o totale a destra) se: \\

$\forall y \in B \ldotp \exists x \in A : y = f(x)$ \\

\noindent
(equivalentemente: $im(f) = codom(f)$)


\subsubsection{Biiettività}

Una funzione si dice \textbf{biiettiva} (o biiezione, o anche corrispondenza 1 a 1 o biunivoca) se è sia iniettiva che suriettiva. \\

\noindent
Osservazione: \\

$f : A \rightarrow B$ è biiettiva $\implies \left | A \right | = \left | B \right |$ \\

\noindent
$A$ e $B$ possono essere infiniti. \\

\noindent
$\left | X \right | < \left | Y \right | \iff \exists$ una funzione iniettiva $X \rightarrow Y \land \nexists$ una biiezione $X \rightarrow Y$.


\subsection{Definizione di Immagine}

L'insieme di tutti i valori di $f : A \rightarrow B$ valutata in ogni elemento di un insieme $S \subseteq A$ si dice l'immagine di S tramite $f$: \\

$f[S] = f(S) := \braces{f(s) \in B : s \in S \subseteq A} \subseteq B$ \\

Im$f = im(f) = f[A]$ \\

\noindent
L'immagine del dominio di una funzione $f$ tramite $f$ si dice immagine di $f$.\\

\noindent
Il valore di $f : A \rightarrow B$ valutata in $x \in A$ si dice immagine di $x$ tramite $f$. \\

\noindent
$im(f) \subseteq codom(f)$


\subsection{Definizione di Controimmagine}

L'insieme degli elementi del dominio di una funzione $f : A \rightarrow B$ che $f$ associa a tutti gli elementi di $S$ si dice controimmagine, preimmagine o immagine inversa di $S$ tramite $f$: \\

$f^{-1}[S] = f^{-1}(S) := \braces{x \in A : f(x) \in S \subseteq B} \subseteq A$


\subsection{Definizione di Restrizione}

Detta anche restrizione del dominio o restrizione a sinistra. \\

\noindent
$f : A \rightarrow B, X \subseteq A$ \\

\noindent
Si dice restrizione di $f$ ad $X$ la funzione: \\

$f_X : X \rightarrow B$ \\
\indent
$f_X(x) = f(x)$ $\forall x \in X$ \\

\noindent
O equivalentemente: \\

$f_X : X \rightarrow B = \braces{(a, b) \in f : a \in X}$ \\

\noindent
O ancora: \\

$f_X : X \rightarrow B = f \circ i$ \\

\noindent
Dove $i : X \rightarrow A$ è l'inclusione di $X$ in $A$ data da $i(a) = a$.


\subsection{Definizione di Troncatura}

Detta anche corestrizione, restrizione del codominio o restrizione a destra. \\

\noindent
Data $f : A \rightarrow B \land Im(f) \subseteq Y$, si dice \textbf{troncatura} di $f$ ad $Y$ la funzione: \\

$f^Y : A \rightarrow Y = \braces{(a, y) \in A \times Y : y = f(a)}$ \\

\noindent
Osservazione: \\
Il codominio viene ristretto.\\

\noindent
In generale prima si restringe e poi si tronca una funzione.


\subsection{Composizione}

L'elemento $b$ che compone in $(g \circ f)$ è unico $\forall a \in A$.



\section{Strutture Algebriche}


\subsection{Definizione di Operazione Binaria}

Sia $U$ un insieme. Si dice \textbf{operazione binaria} una funzione $o : U \times U \rightarrow U$.


\subsection{Definizione di Struttura Algebrica}

Una \textbf{struttura algebrica} è una n-upla data da un insieme ed una o più operazioni su di esso: \\

$(U, o)$


\subsection{Definizione di Associatività}

Si dice che $*$ è associativa se $\forall a,b,c \in A \ldotp a * (b * c) = (a * b) * c$.


\subsection{Definizione di Elemento Neutro}

Sia $(A, *)$ un insieme con una operazione binaria (magma): \\

$* : A \times A \rightarrow A$ \\
\indent
$(a, b) = a * b$ \\

\noindent
Si dice che $e \in A$ è un \textbf{elemento neutro} per $*$ se: \\

$\forall a \in A \ldotp e * a = a * e = a$


\subsection{Definizione di Inverso, Inverso Destro e Inverso Sinistro}

Se $(X, *)$ ammette elemento neutro $e$ si dice che $\forall x \in X$: \\

$x'$ è inverso destro di $x$ se $\exists x' \in X : x * x' = e$ \\
\indent
$x''$ è inverso sinistro di $x$ se $\exists x'' \in X : x'' * x = e$ \\
\indent
$x'''$ è inverso di $x$ se $x'''$ è inverso destro di $x$ $\land$ $x'''$ è inverso sinistro $x$.


\subsection{Definizione di Commutatività}

Si dice che $*$ è commutativa se $\forall a,b \in A \ldotp a * b = b * a$.


\subsection{Definizione di Monoide}

$(A, *)$ (magma) è detto \textbf{monoide} se $*$ è associativa (semigruppo) e ammette elemento neutro.


\subsection{Definizione di Gruppo}

$(A, *)$ è detto \textbf{gruppo} se è un monoide $\land$ ogni elemento di $A$ ammette inverso (necessariamente unico, destro e sinistro, solitamente indicato con $a^{-1}$).


\subsection{Definizione di Gruppo Abelliano}

$(A, *)$ è detto \textbf{gruppo abelliano} o commutativo se oltre ad essere un gruppo, $*$ è commutativa.


\subsection{Definizione di Sottrazione}

La sottrazione è definita come somma con l'opposto di un elemento in $(\Z, +)$. \\

$a - b = a + (-b)$



\section{Definizione di Successione}

Una funzione $f$ si dice successione se: \\

$f : \N \rightarrow A$



% TODO: TEOREMA: Se la funzione inversa esiste è unica.
% TODO: TEOREMA: f è invertibile /iff f é biiettiva
% TODO: PROPOSIZIONE: La composizione di funzioni è associativa



\section {Matrici}

Matrice $m \times n$ a coefficienti in $\K$: $Mat_{m,n}(\K) = \K^{m,n}$ \\

\noindent
Elementi $a_{i, j}$ \\

\noindent
$m \neq n \rightarrow$ matrice rettangolare \\
$m = n \rightarrow$ matrice quadrata \\

\noindent
$A \in Mat_{m,n}(\K)$ \\

\noindent
$\backslash$ = diagonale principale \\
$/$ = diagonale secondaria \\


\subsection{Matrici quadrate particolari}


\subsubsection{Triangolare superiore}

$a_{i,j} = 0$ $\forall i > j$ \\

\noindent
$\begin{pmatrix}
a & b & c \\
0 & d & e \\
0 & 0 & f
\end{pmatrix}$


\subsubsection{Triangolare inferiore}

$a_{i,j} = 0$ $\forall j > i$ \\

\noindent
$\begin{pmatrix}
a & 0 & 0 \\
b & c & 0 \\
d & e & f
\end{pmatrix}$


\subsubsection{Diagonale}

$a_{i,j} = 0$ $\forall i > j$ \\

\noindent
$\begin{pmatrix}
a & 0 & 0 \\
0 & b & 0 \\
0 & 0 & c
\end{pmatrix}$


\subsubsection{Scalare (Diagonale)}

con $a_{i,i} = k \in \K$ \\

\noindent
$\begin{pmatrix}
a & 0 & 0 \\
0 & a & 0 \\
0 & 0 & a
\end{pmatrix}$


\subsubsection{Identica (o identità) di ordine $n$ (Scalare con $k = 1$)}

$a_{i,i} = k$ \\

\noindent
$I_n$ \\

\noindent
$I_3 = \begin{pmatrix}
1 & 0 & 0 \\
0 & 1 & 0 \\
0 & 0 & 1
\end{pmatrix}$


\subsubsection{Nulla $\underline{0}$}

$a_{i,j} = 0$ $\forall i, j$ \\

\noindent
$\begin{pmatrix}
0 & 0 \\
0 & 0
\end{pmatrix}$


\subsection{Matrice Trasposta}

$A \in Mat_{m,n}(\K)$ \\

\noindent
Matrice trasposta di A: \\
$A^T$ \\

\noindent
Righe e colonne scambiate. \\
$a_{i,j} = a{j,i}$ \\

\noindent
$A = (A^T)^T$ \\

\noindent
$A = A^T \implies A$ è simmetrica, $A$ è quadrata


\subsection{Somma tra matrici}

Somma elemento per elemento (per matrici di dimensioni uguali) \\

\noindent
$(Mat_{m,n}(\K), +)$ è un gruppo abelliano.


\subsection{Prodotto per scalare}


\subsubsection{Proprietà}

Distributivo rispetto all'addizione


\subsection{Prodotto righe per colonne}


\subsubsection{Proprietà}

Non commutativo \\
Associativo \\
Distributivo rispetto alla somma \\

\noindent
$A \cdot B = \underline{0} \centernot\implies A = \underline{0} \lor B = \underline{0}$ \\

\noindent
$(A \cdot B)^T = B^T \cdot A^T$


\subsection{Calcolo del determinante}

Solo per matrici quadrate. \\

\noindent
$\left | A \right |$ \\

\noindent
$det(A)$


\subsubsection{$2 \times 2$}

Differenza prodotto diagonali


\subsubsection{$3 \times 3$ (Sarrus)}

Differenza (somme prodotti diagonali e prodotti sovradiagonali). \\

\noindent
Se $A$ è triangolare superiore il determinante è il prodotto della diagonale


\subsubsection{Regola di Laplace}

$A \in Mat_n(\K),$ $n \geq 2$ \\

\noindent
$\left | A \right | = \sum_{j=1}^{n} (-1)^{i+j} a_{i,j} \cdot \left | A_{i,j} \right |$ \\

\noindent
Dove $A_{i,j}$ è la matrice ottenuta da $A$ togliendo ad $A$ la $i$-esima riga e la $j$-esima colonna. \\

\noindent
Il valore $(-1)^{i+j} \left | A_{i,j} \right |$ è detto complemento algebrico di $a_{i,j}$. \\

\noindent
Osservazione: \\
Il termine $(-1)^{i+j}$ indica che se la somma degli indici di riga e colonna è dispari, il segno nella somma va cambiato, altrimenti va mantenuto.

\noindent
Osservazione: \\
Si può applicare Laplace per righe / colonne qualsiasi, ma per snellire i conti conviene scegliere righe / colonne con il maggior n° di $0$.


\subsubsection{Proprietà dei Determinanti}

$\left | I_n \right | = 1$ \\

\noindent
$\left | A \right | = \prod_{i=1}^n a_{i,i}$ \\

\noindent
Quando $A$ è triangolare / diagonale (anche rispetto alla diagonale secondaria, anche se in quel caso non si chiama triangolare / diagonale) \\

\noindent
$\left | A \right | = \left | A^T \right |$ \\

\noindent
$\left | A \cdot B \right | = \left | A \right | \cdot \left | B \right |$ \\

\noindent
Osservazione: \\
In generale non vale per la somma. \\

\noindent
Se in $A$ c'è una riga / colonna nulla, allora $\left | A \right | = 0$ \\

\noindent
Scambiando righe e colonne % TODO (trasposta?)
il determinante cambia di segno. \\

\noindent
Se una riga / colonna è combinazione lineare di altre righe / colonne, allora $\left | A \right | = 0$ e viceversa.


\subsubsection{Definizione di Combinazione Lineare}

Quando una riga / colonna si può scrivere utilizzando le altre righe / colonne combinate solo con operazioni di somma / prodotto e/o prodotto per scalare. \\

\noindent
Osservazione: \\
Se una riga / colonna è multipla di un'altra riga / colonna allora è una sua combinazione lineare. \\

\noindent
Sommando a una riga / colonna una combinazione lineare delle altre righe / colonne il determinante non cambia. \\


\subsection{Definizione di Matrice Singolare}
Una matrice quadrata si dice non singolare se il suo deteminante è $\neq 0$. Altrimenti si dice singolare.


\subsection{Definizione di Matrice Inversa}

Si dice inversa di $A$, se $\exists$, la matrice $A^{-1}$ tale che: \\

\noindent
$A \cdot A^{-1} = A^{-1} \cdot A = I_n$ \\

\noindent
Osservazione: \\
Sia $A \in Mat_n(\K),$ $\exists A^{-1} \iff \left | A \right | \neq 0$ \\

\noindent
Cioè $A$ ammette inversa se e solo se $A$ è non singolare.


\subsubsection{Calcolo della Matrice Inversa (Metodo del Complemento Algebrico)}

Data $A = (a_{i,j}) \in Mat_n(\K)$ si dice aggiunta di $A$ la matrice $A_a \in Mat_n(\K)$ ottenuta sostituendo in $A$ ogni elemento col suo complemento algebrico ($c$). \\

\noindent
$c_{i,j} = (-1)^{i+j} \left | A_{i,j} \right |$ \\

\noindent
$\left | A \right | \neq 0 \implies A^{-1} = \frac{1}{\left | A \right |} \cdot A_a^T$


\subsection{Rango}


\subsubsection{Definizione di Minore di Ordine $p$}

Data una matrice $A \in Mat_{m,n}(\K)$ si dice minore di ordine $p$ una matrice quadrata di ordine $p$ ottenuta da $A$ sopprimendo $n-p$ colonne e $m-p$ righe.

\subsubsection{Definizione di Rango}

Data una matrice $A \in Mat_{m,n}(\K)$ dire che il rango di $A$ è $p$: \\

\noindent
$rg(A) = p$ \\
$r(A) = p$ \\
$\rho(A) = p$ \\

\noindent
con $p \leq min(m, n)$ \\

\noindent
significa dire che $A$ ha un minore non singolare di ordine $p$, e che ogni eventuale minore di ordine $p + 1$ è singolare.

\noindent
$r(A) = 0 \iff A = \underline{0}$ \\

\noindent
Se $A \in Mat_n(\K)$ allora $r(A) = n \iff \left | A \right | \neq 0$ \\

\noindent
$A$ ha rango massimo $= A \in Mat_n(\K),$ $r(A) = n$ \\

\noindent
$1 \leq rg(A) \leq min(m,n),$ $A \in Mat_{m,n}(\K),$ $A \neq \underline{0}$


\subsubsection{Teorema degli Orlati (Teorema di Kronecker)}

$A \in Mat_{m,n}(\K)$.\\
Il rango di $A$ è $p \iff \exists$ in $A$ un minore di ordine $p$ ($M_p$) non singolare $\land$ ogni minore di ordine $p + 1$ che contiene completamente $M_p$ è singolare.

\subsubsection{Definizione di Contiene Completamente}
Che ha al suo interno.

\end{document}

% Fonti:

% - Miei appunti Andrea Ferraguti
% - Miei appunti Luca Giuzzi
% - Miei appunti Lara Ercoli
% - Appunti Davide
% - https://www.youmath.it/formulari/formulari-insiemistica/1586-cardinalita-di-un-insieme.html#:~:text=La%20cardinalit%C3%A0%20di%20un%20insieme,elementi%20che%20costituiscono%20l%27insieme.
% - https://it.wikipedia.org/wiki/Relazione_(matematica)
% - https://it.wikipedia.org/wiki/Relazione_totale
% - https://en.wikipedia.org/wiki/Function_(mathematics)
% - https://it.wikipedia.org/wiki/Funzione_(matematica)
% - https://it.wikipedia.org/wiki/Funzione_iniettiva
% - https://it.wikipedia.org/wiki/Funzione_suriettiva
% - https://it.wikipedia.org/wiki/Corrispondenza_biunivoca
% - https://en.wikipedia.org/wiki/Restriction_(mathematics)
% - https://en.wikipedia.org/wiki/Magma_(algebra)
% - https://it.wikipedia.org/wiki/Inverso_destro_e_sinistro
% - https://en.wikipedia.org/wiki/Semigroup
% - https://en.wikipedia.org/wiki/Monoid
% - https://en.wikipedia.org/wiki/Group_(mathematics)
% - https://en.wikipedia.org/wiki/Abelian_group
