\documentclass[a4paper, twoside, english, 11pt]{book}

\usepackage{amsmath}   % for implies
\usepackage{mathtools} % for coloneqq

\newcommand{\braces}[1] {\left \{ #1 \right \}}

\newcommand{\C}{\mathcal C}
\newcommand{\D}{\mathcal D}
\newcommand{\I}{\mathcal I}
\newcommand{\catname}[1]{{\normalfont\textbf{#1}}}
\newcommand{\Set}{\catname{Set}}
\newcommand{\Cat}{\catname{Cat}}

% TODO Hom & Ob

\begin{document}

\pagestyle{headings}

\frontmatter

\begin{titlepage}
	\begin{center}
		{\huge \bfseries Category Theory Course Notes \\}
		\vspace{1.5cm}
		{\Large \bfseries Ettore Forigo}
	\end{center}
\end{titlepage}

\mainmatter

\chapter{}



\section{Definition of Category}

A \textbf{category} (1-category) $\C$ consists of: \\

1 - A class $Ob(\C)$ of objects of $\C$ \\

2 - $\forall X, Y \in Ob(\C) \ldotp$ \\
\indent\indent
a class $Hom_\C(X, Y)$ of \textbf{morphisms} from $X$ to $Y$ \\

3 - $\forall X \in Ob(\C) \ldotp$ \\
\indent\indent
an \textbf{identity morphism} $id_X \in Hom_\C(X, X)$ \\

4 - $\forall X, Y, Z \in Ob(\C) \ldotp$ \\
\indent\indent
a \textbf{composition rule}: \\\\
\indent\indent\indent
$Hom_\C(Y, Z) \times Hom_\C(X, Y) \rightarrow Hom_\C(X, Z)$ \\
\indent\indent\indent
$(g, f) \mapsto g \circ f$ \\\\

\noindent
Such that it satisfies the following axioms: \\

1 - \textbf{Associativity of composition}: \\\\
\indent\indent
$\forall X, Y, Z, W \in Ob(\C) \ldotp$ \\
\indent\indent
$\forall f \in Hom_\C(X, Y), g \in Hom_\C(Y, Z), h \in Hom_\C(Z, W) \ldotp$ \\
\indent\indent
$h \circ (g \circ f) = (h \circ g) \circ f$ \\

2 - \textbf{Neutrality}: \\\\
\indent\indent
$\forall X, Y \in Ob(\C) \ldotp$ \\
\indent\indent
$\forall f \in Hom_\C(X, Y) \ldotp$ \\
\indent\indent
$id_Y \circ f = f \land f \circ id_X = f$ \\



\section{Thin Categories}

A category is \textbf{thin} if parallel morphisms are always the same, meaning that there is only one morphism between two objects. \\

In a thin category all morphisms are monic and epic.



\section{Definition of Initial Object}

An object $I$ of a category $\C$ is \textbf{initial} (dual of terminal, special case of a colimit (of a functor from $\C$ to the empty category)) \\
\indent
$\Updownarrow$ \\ % \iff
$\forall X \in Ob(\C) \ldotp$ \\
$\exists! f \in Hom_\C(I, X)$



\section{Definition of Terminal Object}

An object $T$ of a category $\C$ is \textbf{terminal} (dual of initial, special case of limit (of a functor from the empty category to $\C$)) \\
\indent
$\Updownarrow$ \\ % \iff
$\forall X \in Ob(\C) \ldotp$ \\
$\exists! f \in Hom_\C(X, T)$



\section{Definition of Monomorphism}

A morphism $f : X \rightarrow Y$ in a category $\C$ ($f \in Hom_\C(X, Y)$) is a \textbf{monomorphism} (or monic in $\C$) (dual of epimorphism) \\
\indent
$\Updownarrow$ \\ % \iff
$\forall Z \in Ob(\C) \ldotp \forall p, q \in Hom_\C(Z, X) \ldotp$ \\
$f \circ p = f \circ q \implies p = q$ \\

\noindent
Example:

In $\Set$ monomorphisms are precisely the injective maps. \\

\noindent
Monomorphisms ``can be cancelled'' from the left.



\section{Definition of Split Monomorphism}

A \textbf{split monomorphism} (dual of split epi) is a morphism $f : X \rightarrow Y$ such that there exists a morphism $g : Y \rightarrow X$ such that: \\

$g \circ f = id_X$ \\

\noindent
Proposition: every split mono is a mono. \\
Proposition: in $\Set$, every mono $f : X \rightarrow Y$ where $X$ is inhabited is a split mono, assuming LEM holds.



\section{Definition of Epimorphism}

A morphism $f : X \rightarrow Y$ in a category $\C$ ($f \in Hom_\C(X, Y)$) is an \textbf{epimorphism} (or epic in $\C$) (dual of monomorphism) \\
\indent
$\Updownarrow$ \\ % \iff
$\forall Z \in Ob(\C) \ldotp \forall p, q \in Hom_\C(Y, Z) \ldotp$ \\
$p \circ f = q \circ f \implies p = q$ \\

\noindent
Example:

In $\Set$ epimorphisms are precisely the surjective maps. \\

\noindent
Epimorphisms ``can be cancelled'' from the right.



\section{Definition of Split Epimorphism}

A \textbf{split epimorphism} (dual of split mono) is a morphism $f : X \rightarrow Y$ such that there exists a morphism $g : Y \rightarrow X$ such that: \\

$f \circ g = id_Y$ \\

\noindent
Proposition: every split epi is an epi. \\
Proposition: in $\Set$, every epi is a split epi $\iff$ assuming LEM holds.



\section{Definition of Isomorphism}

A morphism $f : X \rightarrow Y$ in a category $\C$ ($f \in Hom_\C(X, Y)$) is an \textbf{isomorphism} \\
\indent
$\Updownarrow$ \\ % \iff
$\exists g \in Hom_\C(Y, X) \ldotp$ \\
$f \circ g = id_Y \land g \circ f = id_X$ \\

\noindent
$id_X \forall X \in Ob(\C)$ is always an isomorphisms for every category $\C$.

\noindent
Objects $X$ and $Y$ in a category $\C$ are \textbf{isomorphic} \\
\indent
$\Updownarrow$ \\ % iff
there exists an isomorphism between $X$ and $Y$ ($X \cong Y$) \\

\noindent
In $\Set$, if there exists an isomorphism between $X$ and $Y$, $X$ and $Y$ are called eqinumerous.



\section{Definition of Opposite Category}

``The mother of all dualities'' \\

\noindent
Let $\C$ be a category. Then its opposite category $\C^{op}$ is the following category: \\

- $Ob(\C^{op}) \coloneqq Ob(\C)$ \\

- $Hom_{\C^{op}}(X, Y) \coloneqq Hom_\C(Y, X)$ \\

- identities and composition inherited from $\C$ \\
\indent\indent
$id_X \in Hom_\C(X, X) = id_X^{op} \in Hom_{\C^{op}}(X, X)$ \\
\indent\indent
$f \circ g \coloneqq g^{op} \circ f^{op}$ \\\\

\noindent
Observations / Remarks: \\

- An object $I$ of $\C$ is initial in $\C$

\indent\indent
$\Updownarrow$ % \iff

\indent
I is terminal when regarded as an object of $\C^{op}$ \\

- A morphism in $\C$ is a monomorphism

\indent\indent
$\Updownarrow$ % \iff

\indent
it is an epimorphism in $\C^{op}$ \\



\section{Dualities?}



\section{Definition of Product}

A \textbf{product} (special case of limit) of two objects $X$ and $Y$ in a category $\C$ consists of: \\

- an object $P$ of $\C$ \\

- a morphism $\pi_X : P \rightarrow X$ in $\C$ \\

- a morphism $\pi_Y : P \rightarrow Y$ in $\C$ \\

\noindent
such that for every object $Q$ of $\C$ together with morphisms $\varphi_X : Q \rightarrow X, \varphi_Y : Q \rightarrow Y$ there is exactly one morphism $Q \rightarrow P$ such that the following diagram commutes: \\

$\varphi_X = \pi_X \circ !$ \\
\indent
$\varphi_Y = \pi_Y \circ !$ \\

\noindent
Remarks:

- Products are always associative and commutative up to isomorphism.

- There is also the notion of the (co) product of zero, one, three, four, ... objects.

- The zero case of a product is just a terminal object, an object with exactly one morphism from each object.



\section{Definition of Coproducts}

A \textbf{coproduct} (special case of colimits) of two objects $X$ and $Y$ in a category $\C$ consists of: \\

- an object $C$ of $\C$ \\

- a morphism $\iota_X : X \rightarrow C$ in $\C$ \\

- a morphism $\iota_Y : Y \rightarrow C$ in $\C$ \\

\noindent
such that for every object $D$ of $\C$ together with morphisms $\chi_X : X \rightarrow D, \chi_Y : Y \rightarrow D$ there is exactly one morphism $C \rightarrow D$ which renders the following diagram commutative: \\

$\chi_X = ! \circ \iota_X$ \\
\indent
$\chi_Y = ! \circ  \iota_Y$ \\

\noindent
Remarks: \\

- Products in $\C^{op}$ are precisely coproducts in $\C$ \\

- The zero case of a coproduct is the same as an initial object.



\section{Definition of Functor}

A (covariant) \textbf{functor} $F : \C \rightarrow \D$ from a category $\C$ to a category $\D$ consists of: \\

- an object $F(X) \in Ob(\D)$ for each object $X \in Ob(\C)$ \\

- a morphism $F(f) : F(X) \rightarrow F(Y)$ in $\D$ for each morphism $f : X \rightarrow Y$ in $\C$ \\

\noindent
such that: \\

- $\forall X \in Ob(\C) \ldotp F(id_X) = id_{F(X)}$ \\

- $\forall X, Y, Z \in Ob(\C) \ldotp \forall f : X \rightarrow Y \in \C, g : Y \rightarrow Z \text{ in } \C \ldotp F(g \circ f) = F(g) \circ F(f)$ \\

\noindent
Motto: \\
Functors $\I \rightarrow \C$ are $\I$-shaped \textbf{diagrams} in $\C$



\section{Definition of Contravariant Functor}

A \textbf{contravariant functor} $\C \rightarrow \D$ is a covariant functor $\C^{op} \rightarrow \D$



\section{Forgetful Functors?}



\section{Powerset Functor[s?]?}



\section{Definition of Discrete Category}

The \textbf{discrete category} associated with a set $X$, written $\D(X)$, is a category containing all the objects of $X$ as objects, and no morphisms between different objects, just the identity morphisms.



\section{Definition of Induced Functors}

Claim: \\
Any map between sets can be turned into a functor. \\

\noindent
Let $f : X \rightarrow Y$ be a map between sets. \\

\noindent
Consider the discrete categories $\D(X), \D(Y)$. \\

\noindent
Then $f$ induces the following functor $\D(X) \rightarrow D(Y)$: \\
\indent
$x \mapsto f(x)$ \\
\indent
$id_x \mapsto id_{f(x)}$



\section{Definition of the Walking Arrow?}



\section{Definition of the Walking Commutative Triagle?}

% TODO mention degenerate triangle case?



\section{Definition of Essentially Surjective Functor}

A functor $F : \C \rightarrow \D$ is \textbf{essentially surjective} iff: \\

$\forall Y \in Ob(\D) \ldotp \exists X \in Ob(\C) | F(X) \cong Y$



\section{Definition of Faithful Functor}

A functor $F : \C \rightarrow \D$ is \textbf{faithful} iff: \\

$\forall X, Y \in Ob(\C) \ldotp$ \\
\indent
$\forall f, g : X \rightarrow Y$ in $\C$ \\
\indent
$F(f) = F(g) \implies f = g$ \\

\noindent
Reformulation: iff \\

$\forall X, Y \in Ob(\C) \ldotp$ \\
\indent
$Hom_\C(X, Y) \rightarrow Hom_\D(F(X), F(Y))$ \\
\indent
$f \mapsto F(f)$ \\

\noindent
is injective.




\section{Definition of Full Functor}

A functor $F : \C \rightarrow \D$ is \textbf{full} iff: \\

$\forall X, Y \in Ob(\C) \ldotp$ \\
\indent
$\forall g : F(X) \rightarrow F(Y)$ in $\D$ \\
\indent
$\exists f : X \rightarrow Y$ in $\C | F(f) = g$ \\

\noindent
Reformulation: iff \\

$\forall X, Y \in Ob(\C) \ldotp$ \\
\indent
$Hom_\C(X, Y) \rightarrow Hom_\D(F(X), F(Y))$ \\
\indent
$f \mapsto F(f)$ \\

\noindent
is surjective.



\section{Definition of Fully Faithful Functor}

A functor is \textbf{fully faithful} iff it is full and faithful. \\

\noindent
Reformulation: iff \\

$\forall X, Y \in Ob(\C) \ldotp$ \\
\indent
$Hom_\C(X, Y) \rightarrow Hom_\D(F(X), F(Y))$ \\
\indent
$f \mapsto F(f)$ \\

\noindent
is bijective.



\section{Definition of Elementary Equivalence}

An \textbf{elementary equivalence} is a fully faithful, essentially surjective functor.



\section{Definition of Equivalence of Categories}

Categories are called \textbf{equivalent} iff there is an elementary equivalence between them. \\

\noindent
Remark:
Equivalent categories have exactly the same categorical properties.



\section{Definition of Natural Transformation}

A \textbf{natural transformation} $\eta : F \Rightarrow G$ between two functors $F, G : C \rightarrow D$ consists of: \\

- for each object $X \in Ob(\C)$ a morphism $\eta_X : F(X) \rightarrow G(X)$ in $\D$ \\

\noindent
such that for all morphisms $f : X \rightarrow Y$ in $\C$, the \textbf{naturality square} commutes: \\

$G(f) \circ \eta_X = \eta_Y \circ F(f)$ \\

\noindent
Motto: \\
Natural transformations are \textbf{uniform} families of morphisms.



\section{Definition of Functor Category}

Let $\C, \D$ be categories. \\
The \textbf{functor category} $[\C, \D]$ has: \\

- as objects: all functors $\C \rightarrow \D$ \\

- as morphisms: $Hom_{[\C, \D]}(F, G) \coloneqq \braces{h : F \Rightarrow G | h \text{ is a natural transformation}}$ \\

- as identity: for the object F, the identity $id_F : F \Rightarrow F \\
\indent\indent
(id_F)_X : F(X) \rightarrow F(X)$ \\
\indent\indent
given by $id_{F(X)}$ \\

- as composition rule: \\
\indent\indent
$(\omega \circ \eta)_X \coloneqq \omega_X \circ \eta_X$ \\

\indent\indent
$\omega_X : G(X) \rightarrow H(X)$ \\
\indent\indent
$\eta_X : F(X) \rightarrow G(X)$ \\

and $\omega \circ \eta$ should be natural.



\section{Definition of Small Category}

A category $\C$ is small when $Ob(\C)$ is just a set and not a proper class.



\section{Definition of Category of Categories}

The \textbf{1-category of 1-categories}, $\Cat$ has: \\

- as objects: all categories \\

- as morphisms: $Hom_\Cat(\C, \D) \coloneqq \braces{F : \C \rightarrow \D | F \text{ is a functor}}$ \\

- as identities $Id_F$ (the identity functor?) \\

- as composition rule: \\
\indent\indent
$F : \C \rightarrow \D$ \\
\indent\indent
$G : \D \rightarrow \catname E$ \\

\indent\indent
$G \circ F : \C \rightarrow \catname E$ \\
\indent\indent
$X \mapsto G(F(X))$ \\
\indent\indent
$f \mapsto G(F(f))$ \\\\

\noindent
There are two issues with this definition: \\

- Size issue (in ZFC). (it's too big, the objects don't fit in a proper class?) \\
\indent\indent
Remedies: \\

\indent\indent
- just consider the category of small categories \\

\indent\indent
- switch foundations \\

- It ignores natural transformations \\
\indent\indent
Remedy: \\
\indent\indent
Consider the 2-category of 1-categories \\

\indent\indent
The 2-category of 1-categories has: \\

\indent\indent\indent
- as objects: all 1-categories \\

\indent\indent\indent
- as morphisms: functors \\

\indent\indent\indent
- as -2-morphisms / 2-cells: natural transformations \\



\section{Definition of Cone}

A \textbf{cone} of a diagram (functor) $F : \I \rightarrow \C$ in a category $\C$ consists of: \\

- an object A of $\C$ (the "tip" of the cone) \\

- for each object $X \in Ob(\C)$, a morphism $\pi_X : A \rightarrow F(X)$ \\

\noindent
such that for all morphisms $f: X \rightarrow Y$ in $\I$, the triangle: \\

$\pi_Y = \pi_X \circ F(f)$ //

\noindent
commutes.



\section{Definition of Cocone}

A \textbf{cocone} of a diagram (functor) $F : \I \rightarrow \C$ in a category $\C$ consists of: \\

- an object A of $\C$ (the "tip" of the cocone) \\

- for each object $X \in Ob(\C)$, a morphism $\pi_X : F(X) \rightarrow A$ \\

\noindent
such that for all morphisms $f: X \rightarrow Y$ in $\I$, the triangle: \\

$\pi_X = \pi_Y \circ F(f)$ //

\noindent
commutes.



\section{Definition of Morphism Between Cones}

A \textbf{morphism} between a cone $(A, (\pi_X)_X)$ and a further cone $(B, (\phi_X)_X)$ of a diagram $F : \I \rightarrow \C$ consists of a morphism $f : A \rightarrow B$ in $\C$ such that: \\

$\pi_X = \pi_Y \circ f$



\section{Definition of Limit}

A \textbf{limit} of a diagram $F : \I \rightarrow \C$ is a \textbf{terminal cone} of $F$, that is, a terminal object in the category of of cones of cones of $F$.



\section{Definition of Colimit}

A \textbf{colimit} of a diagram $F : \I \rightarrow \C$ is an \textbf{initial cocone} of $F$.



\section{Definition of Equalizer of Two Set-Theoretic Maps}

Let $f, g : X \rightarrow Y$. Then the \textbf{equalizer} of $f$ and $g$ is the following function: \\

$Eq(f, g) = {x \in X | f(x) = g(x)}$



\section{Definition of Pullback}



\section{Definition of Pushout}



\section{Definition of Small Diagram}

A \textbf{small diagram} in $\C$ is a diagram $\I \rightarrow \C$ where $\I$ is a small category.



\section{Definition of Complete Cateogory}

A category $\C$ is \textbf{complete} iff every small diagram in $\C$ has a limit (it has all small limits).



\section{Definition of Cocomplete Category}

A category $\C$ is \textbf{cocomplete} iff every small diagram in $\C$ has a colimit (it has all small colimits). \\

$\C$ complete $\iff \C^{op}$ cocomplete.



\section{Formula for Limits in Set}
\section{Formula for Colimits in Set}
\section{Definition of Yoneda Lemma}
\section{Definition of Presheaf}
\section{Definition of Representable Presheaf}
\section{Yoneda Embedding}
\section{Yoneda Style Proofs}
\section{Definition of Adjoint Functors}
\section{Currying Adjunction}
\section{Adjunction of Logical Connectives}
\section{Monoids}
\section{Monoids Categorically}
\section{Monoidal Categories}
\section{Monads}

\end{document}
