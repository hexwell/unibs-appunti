\documentclass[a4paper, twoside, english, 11pt]{book}

\usepackage{amsmath}   % for implies

\newcommand{\braces}[1] {\left \{ #1 \right \}}

\DeclareMathOperator{\dom}{dom}
\DeclareMathOperator{\codom}{codom}
\DeclareMathOperator{\Ima}{Im}

\newcommand{\C}{\mathcal C}
\newcommand{\catname}[1]{{\normalfont\textbf{#1}}}
\newcommand{\Set}{\catname{Set}}

\begin{document}

\pagestyle{headings}

\frontmatter

\begin{titlepage}
	\begin{center}
		{\huge \bfseries Category Theory Course Exercises \\}
		\vspace{1.5cm}
		{\Large \bfseries Ettore Forigo}
	\end{center}
\end{titlepage}

\mainmatter

\chapter{}



\section{Proof}

Proof: $f$ is an epimorphism in $\Set \iff f$ is surjective. \\\\

\noindent
Definition of surjectivity: \\
\indent
$\forall y \in \codom(f) \ldotp \exists x \in \dom(f) | y = f(x)$ \\

\noindent
Definition of epimorphism: \\
\indent
$f : X \rightarrow Y, p,q : Y \rightarrow Z$ \\
\indent
$p \circ f = q \circ f \implies p = q$ \\

\noindent
To show: \\

$f$ is surjective $\implies f$ is an epimorphism \\\\

\noindent
Let $f$ be surjective. \\

\noindent
Claim: \\

$f$ is an epimorphism. \\\\

\noindent
To show: \\

\noindent
$\forall x \in \dom(f) \ldotp p(f(x)) = q(f(x)) \implies \forall y \in \dom(p) = \dom(q) \ldotp p(y) = q(y)$ \\

\noindent
Using: \\

\noindent
$\Ima(f) = \dom(p) = \dom(q)$ \\
\indent\indent\indent
$\Downarrow$ \\ % \implies
$\forall x \in \dom(f) \ldotp p(f(x)) = q(f(x)) \implies \forall y \in \dom(p) = \dom(q) \ldotp p(y) = q(y)$ \\

\noindent
To show: \\

$\Ima(f) = \braces{f(x) | x \in \dom(f)} = \braces{y | \exists x \in \dom(f) | y = f(x)}$ \\
\indent\indent\indent
$\stackrel{?}{=}$ \\
\indent
$\codom(f) = dom(p) = dom(q)$ \\

\noindent
True by assumption (surjectivity of $f$). \\

\noindent
$\forall y \in \codom(f) \ldotp \exists x \in \dom(f) | y = f(x) \implies \forall y \in \codom(f) \ldotp y \in \Ima(f)$



\section{Proof}

Proof: Isomorphisms in $\C$ are isomorphisms in $\C^{op}$ \\

\noindent
Definition of isomorphism: \\
\indent
$f : X \rightarrow Y$ is an isomorphism $\iff \exists g : Y \rightarrow X \ldotp f \circ g = id_Y \land g \circ f = id_X$ \\

\noindent
Let $f$ be an isomorphism in $\C$. \\

$\exists g : Y \rightarrow X \in Hom_\C(Y, X) \ldotp f \circ g = id_Y \land g \circ f = id_X$ \\

$f \in Hom_\C(X, Y)$ \\
\indent
$g \in Hom_\C(Y, X)$ \\\\

$f^{op} \in Hom_{\C^{op}}(Y, X)$ \\
\indent
$g^{op} \in Hom_{\C^{op}}(X, Y)$ \\\\

\noindent
Claim: \\

$f^{op}$ is an isomorphism in $\C^{op}$ \\\\

\noindent
To show: \\

$\exists h \in Hom_{\C^{op}}(X, Y) \ldotp f^{op} \circ h = id_X^{op} \land h \circ f^{op} = id_Y^{op}$ \\\\

\noindent
Let $h \in Hom_{\C^{op}}(X, Y)$ be $g^{op}$, the opposite of $g \in Hom_\C(Y, X)$ \\\\

\noindent
To show: \\

$f^{op} \circ g^{op} = id_X^{op} \land g^{op} \circ f^{op} = id_Y^{op}$ \\

\noindent
Using: \\

$id_X \in Hom_\C(X, X) = id_X^{op} \in Hom_{\C^{op}}(X, X)$ \\

$g \circ f = id_X = id_X^{op}$ \\
\indent
$f \circ g = id_Y = id_Y^{op}$ \\\\

\noindent
To show: \\

$f^{op} \circ g^{op} = g \circ f$ \\
\indent
$g^{op} \circ f^{op} = f \circ g$ \\

\noindent
True by definition of composition in opposite category.

\end{document}
